\documentclass[12pt]{article}

\usepackage{geometry}
\usepackage{fancyhdr}
\usepackage{extramarks}
\usepackage{amsmath}
\usepackage{amsthm}
\usepackage{amsfonts}
\usepackage{amssymb}
\usepackage{tikz}
\usepackage[plain]{algorithm}
\usepackage{algpseudocode}
\usepackage{fancyhdr}
\usepackage{array}
\usepackage{wrapfig}
\usepackage{adjustbox}
\usepackage{enumitem}


\newlength{\strutheight}
\settoheight{\strutheight}{\strut}
\newtheorem{theorem}{Theorem}
\renewcommand\qedsymbol{$\blacksquare$}
\newcommand\setitemnumber[1]{\setcounter{enumi}{\numexpr#1-1\relax}}

\geometry{letterpaper,portrait,margin=1in}


\title{\large Abstract Algebra Homework 3}
\author{\large Zachary Meyner}
\date{}

\begin{document}
\maketitle
\begin{enumerate}[label=\textbf{\arabic*}.]
	\setitemnumber{1}
	\item Write the following permutations in cucle notation.
	      \begin{enumerate}
		      \item \begin{equation*}
			            \bigg(\begin{matrix}
				            1 & 2 & 3 & 4 & 5 \\
				            2 & 4 & 1 & 5 & 3
			            \end{matrix}\bigg) = (\begin{matrix}
				            1 & 2 & 4 & 5 & 3)
			            \end{matrix}
		            \end{equation*}
		      \item \begin{equation*}
			            \bigg(\begin{matrix}
				            1 & 2 & 3 & 4 & 5 \\
				            4 & 2 & 5 & 1 & 3
			            \end{matrix}\bigg) = (\begin{matrix}
				            1 & 4)(3 & 5
			            \end{matrix})
		            \end{equation*}
	      \end{enumerate}
	      \setitemnumber{2}
	\item Compute each of the following.
	      \begin{enumerate}
		      \item \begin{equation*}
			            (1 \ 3 \ 4 \ 5)(2 \ 3 \ 4) = (3 \ 4 \ 5 \ 2 \ 1)
		            \end{equation*}
		      \item \begin{equation*}
			            (1 \ 2)(1 \ 2 \ 5 \ 3) = (1 \ 5 \ 2 \ 4 \ 3)
		            \end{equation*}
		      \item \begin{equation*}
			            (1 \ 4 \ 3)(2 \ 3)(2 \ 4) = (1 \ 4 \ 3)(4 \ 3) = (4 \ 3 \ 1)
		            \end{equation*}
	      \end{enumerate}
	      \setitemnumber{3}
	\item Express the following permutations as products of transpositions and identify them as
	      even or odd.
	      \begin{enumerate}
		      \item \begin{equation*}
			            (1 \ 4 \ 3 \ 5 \ 6) = (1 \ 6)(1 \ 5)(1 \ 3)(1 \ 4)
		            \end{equation*}
		            Even transposition
		      \item \begin{equation*}
			            (1\ 5 \ 6)(2 \ 3 \ 4) = (1 \ 6)(1 \ 5)(2 \ 4)(2 \ 3)
		            \end{equation*}
		            Even transposition
		      \item \begin{equation*}
			            (1 \ 4 \ 2 \ 6)(1 \ 4 \ 2) = (1 \ 6)(1 \ 2)(1 \ 4)(1 \ 2)(1 \ 4)
		            \end{equation*}
		            Odd transposition
	      \end{enumerate}
          \setitemnumber{4}
          \item Find $(a_1, a_2, \dots, a_n)^{-1}$.
          \begin{equation*}
              (a_1, a_2, \dots, a_n)^{-1} = (a_n, a_{n-1} \dots, a_2, a_1) = (a_1, a_n, a_{n-1}, \dots, a_2)
          \end{equation*}
          \setitemnumber{13}
          \item Let $\sigma = \sigma_1 \cdots \sigma_m \in S_n$ be the product of disjoint cycles. Prove that the order of $\sigma$ is 
          the least common multiple of the lengths of the cycles $\sigma_1, \dots \sigma_m$.
          \begin{proof}
              Let $\sigma = \sigma_1 \cdots \sigma_m$ where $\sigma_1, \dots, \sigma_m$ are disjoint. Also let $|\sigma| = k$. 
              We know $\sigma_1,\dots, \sigma_m$ commute with each other so \begin{equation*}
                  \sigma^k = \sigma_1^k \cdots \sigma_m^k = id \Longleftrightarrow \sigma_1^k = \cdots = \sigma_m^k = id
              \end{equation*}
              because they are disjoint. This also means that $k$ is a common multiple of $|\sigma_1|, \dots, |\sigma_m|$. 
              Because the order of the cycle is its lenght, the smallest $k$ must then be the least common multiple 
              of the order of the cycles.
          \end{proof}
\end{enumerate}
\end{document}