\documentclass[12pt]{article}

\usepackage{geometry}
\usepackage{fancyhdr}
\usepackage{extramarks}
\usepackage{amsmath}
\usepackage{amsthm}
\usepackage{amsfonts}
\usepackage{amssymb}
\usepackage{tikz}
\usepackage[plain]{algorithm}
\usepackage{algpseudocode}
\usepackage{fancyhdr}
\usepackage{array}
\usepackage{wrapfig}
\usepackage{adjustbox}
\usepackage{enumitem}


\newlength{\strutheight}
\settoheight{\strutheight}{\strut}
\newtheorem{theorem}{Theorem}
\renewcommand\qedsymbol{$\blacksquare$}
\newcommand\setitemnumber[1]{\setcounter{enumi}{\numexpr#1--1\relax}}

\geometry{letterpaper,portrait,margin=1in}


\title{\large Intro to Analysis Homework 4}
\author{\large Zachary Meyner}
\date{}

\begin{document}
\maketitle
\begin{enumerate}
	\item Given that ($a_n$) is a Cauchy sequence, use the definition (as an Epsilon argument) to
	      show that ($a_n^2$) is also a Cauchy sequence. Then provide an example to show that the
	      converse is not true.
	      \begin{proof}
		      Since ($a_n$) is Cauchy iit is also bounded. So $\exists M \in \mathbb{R}$ s.t. $|a_n| \leq M$. Let $\varepsilon > 0$
		      be given. Since ($a_n$) is Cauchy $\exists H \in \mathbb{R}$ s.t. $\forall n,m \geq H |a_n-a_m| < \frac{\varepsilon}{2M}$. \\
		      Then $\forall n,m \geq H$
		      \begin{align*}
			      |a_n^2 - a_m^2| & = |a_n-a_m||a_n+a_m|                       \\
			                      & \leq \frac{\varepsilon}{2M}|a_n+a_m|       \\
			                      & \leq \frac{\varepsilon}{2M}(|a_n| + |a_m|) \\
			                      & = \frac{\varepsilon}{2M}(M+M)              \\
			                      & =\frac{\varepsilon}{2M}(2M)                \\
			                      & = \varepsilon
		      \end{align*}
	      \end{proof}
	      The counterexample to the converse is if ${(a_n)} = {(-1)}^n = \{-1, 1, -1, \cdots \}$ is not Cauchy,
	      but ${(a_n^2)} = {(-1)}^2 = \{1, 1, \cdots\}$. So $(a_n^2)$ is Cauchy, but $(a_n)$ is not.
	\item Given that $(a_n)$ and $(b_n)$ are Cuachy sequences, use the definition (as an Epsilon
	      argument) to show that ($a_{n}b_n$) is also a Cauchy sequence.
	      \begin{proof}
		      Let $\varepsilon > 0$ be given. Because ($a_n$) is Cauchy $|a_n| \leq A$. Since ($a_n$) is Cauchy
		      $\exists H_a \in \mathbb{R}$ s.t. $\forall n,m \geq H_a$. \\
		      $|a_n-a_m| < \frac{\varepsilon}{2B}$ \\
		      Because ($b_n$) is Cauchy $|b_n| \leq B$. Since ($b_n$) is Cauchy $\exists H_b \in \mathbb{R}$ s.t. $\forall n,m \geq H_b$ \\
		      $|b_n-b_m| < \frac{\varepsilon}{2A}$ \\
		      Consider $H = \max(H_a, H_b)$, then $\forall n,m \in \mathbb{R}$
		      \begin{align*}
			      |a_{n}b_n - a_{m}b_m| & = |a_{n}b_n - a_{n}b_m + a_{n}b_m - a_{m}b_m|                                                \\
			                            & \leq |a_{n}b_n - a_{n}b_m| + |a_{n}b_m - a_{m}b_m|         &  & \text{(Triangle Inequality)} \\
			                            & = |a_n||b_n-b_m| + |b_m||a_n-a_m|                                                            \\
			                            & \leq A|b_n-b_m| + B|a_n-a_m|                                                                 \\
			                            & < A(\frac{\varepsilon}{2A}) + B(\frac{\varepsilon}{2B})                                   \\
			                            & = \frac{\varepsilon}{2} + \frac{\varepsilon}{2}                                              \\
			                            & = \varepsilon
		      \end{align*}
	      \end{proof}
          \item Given that $(a_n)$ nad $(b_n)$ are Cauchy sequences, use the definition \\ 
          (an Epsilon argument) to show that $(6a_n-2b_n)$ is also a Cauchy sequence.
          \begin{proof}
              Let $\varepsilon > 0$ be given. Since $(a_n)$ is Cauchy $\exists H_a \in \mathbb{R}$ s.t. $\forall n,m \geq H_a$ \\
              $|a_n-a_m| < \frac{\varepsilon}{12}$ \\
              Since $(b_n)$ is Cauchy $\exists H_b \in \mathbb{R}$ s.t. \\
              $|b_n-b_m| < \frac{\varepsilon}{4}$ \\
              Consider $H = \max(H_a, H_b)$, then $\forall n,m \geq H$
              \begin{align*}
                  |(6a_n-2b_n) - (6a_m-2b_n)| &= |6a_n-6a_m - 2b_n + 2b_m| \\
                  &\leq |6a_n-6a_m| + |-2b_n+2b+m| && \text{(Triangle Inequality)} \\
                  &= 6|a_n-a_m| + 2|b_n-b_m| \\
                  &< 6\frac{\varepsilon}{12} + 2\frac{\varepsilon}{4} \\
                  &= \frac{\varepsilon}{2} + \frac{\varepsilon}{2} \\
                  &= \varepsilon
              \end{align*}
          \end{proof}
\end{enumerate}
\end{document}