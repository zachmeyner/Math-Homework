\documentclass[12pt]{article}

\usepackage{geometry}
\usepackage{fancyhdr}
\usepackage{extramarks}
\usepackage{amsmath}
\usepackage{amsthm}
\usepackage{amsfonts}
\usepackage{amssymb}
\usepackage{tikz}
\usepackage[plain]{algorithm}
\usepackage{algpseudocode}
\usepackage{fancyhdr}
\usepackage{array}
\usepackage{wrapfig}
\usepackage{adjustbox}
\usepackage{enumitem}


\newlength{\strutheight}
\settoheight{\strutheight}{\strut}
\newtheorem{theorem}{Theorem}
\renewcommand\qedsymbol{$\blacksquare$}
\newcommand\setitemnumber[1]{\setcounter{enumi}{\numexpr#1-1\relax}}

\geometry{letterpaper,portrait,margin=1in}


\title{\large Abstract Algebra Homework 2}
\author{\large Zachary Meyner}
\date{}

\begin{document}
\maketitle
\begin{enumerate}[label=\textbf{\arabic*}.]
	\setitemnumber{41}
	\item Prove that
	      \[G=\{a+b\sqrt{2}: a,b \in \mathbb{Q} \ \text{and} \ a \ \text{and} \ b \ \text{are not both zero}\}\]
	      is a subgroup if $\mathbb{R}^*$ under the group operation of multiplication.
	      \begin{proof} Let $n,m \in G \ s.t. \ n = a+b\sqrt{2} \ \text{and} \ m = c+d\sqrt{2}$. \\
            (WTS\@: $n \cdot m \in G$ and $n^{-1} \in G$) \\
            Multiplying $m \cdot n$ we have 
            \begin{align*}
                (a+b\sqrt{2}) \cdot (c+d\sqrt{2}) &= ac+ad\sqrt{2} + bc\sqrt{2} + bd\sqrt{2}^2 && \text{(Distributive Property)} \\
                &= ac + ad\sqrt2 +bc\sqrt2 + 2bd  \\
                &= ac + \sqrt{2}(ad+bc) + 2bd && \text{(Distributive Property)} \\
                &= (ac + 2bd) + (ad + bc)\sqrt{2} && \text{(Commutative and Associative Property)}
            \end{align*}
            and $(ac+2bd) + (ad+bc)\sqrt{2}$ is clearly an in $G$, so $n \cdot m$ must be in $G$.
            Now if we take $n^{-1}$ we get
            \begin{align*}
                \frac{1}{a+b\sqrt{2}} &= \frac{1}{a+b\sqrt{2}} \cdot \frac{(a-b\sqrt{2})}{(a-b\sqrt{2})} && \text{(Multiplying by 1)} \\
                &= \frac{a-b\sqrt{2}}{(a+b\sqrt{2})\cdot(a-b\sqrt{2})} \\
                &= \frac{a-b\sqrt{2}}{a^2-b\sqrt{2}^2} && \text{(Distributive Property)} \\
                &= \frac{a+(-b)\sqrt{2}}{a^2-2b} && \text{(Simplifying)} \\
                &= \frac{a}{a^2-2b} + \frac{-b}{a^2-2b}\sqrt{2} && \text{(Commutative and Associative Propety)} \\
            \end{align*} 
            Because $a$ and $b$ are in $\mathbb{Q}$ we know $\frac{a}{a^2-2b}$ and $\frac{-b}{a^2-2b}$ must also be in $\mathbb{Q}$, so $n^{-1}$ must be in $G$. \\
            $\therefore$ By the 2 step test $G$ is a subgroup of $\mathbb{R}^*$ under the operation of multiplication.
	      \end{proof}
          \pagebreak
    \setitemnumber{45}
    \item Prove that the intersection of two subgroups of a group $G$ is also a subgroup of $G$.
          \begin{proof} Let $P \leq G$ and $H \leq G$. We know that at least the identity element $e \in P \cap H$ Let $a,b \in P \cap H$
            (WTS\@: $ab^{-1} \in P \cap H$). Because $a,b \in P \cap H$ we know
            \begin{align*}
                a &\in P \cap H \\
                \implies a &\in P \ \text{and} \ a \in H \\
                b &\in P \cap H \\
                \implies b &\in P \ \text{and} \ b \in H
            \end{align*}
            Since $P$ and $H$ are subgroups of G we have
            \begin{align*}
                &ab^{-1} \in P \ \text{and} \ ab^{-1} \in H \\
                \implies &ab^{-1} \in P \cap H
            \end{align*}
            Thus $P \cap H$ is also a subgroup of $G$.
          \end{proof}
    \setitemnumber{5}
    \item Find the order of every element in $\mathbb{Z}_{18}$.
    \begin{align*}
    |1|&=18 \\
    |2|&=9 \\
    |3|&=6 \\
    |4|&=9 \\
    |5|&=18 \\
    |6|&=3 \\
    |7|&=18 \\
    |8|&=9 \\
    |9|&=2 \\
    |10|&=9 \\
    |11|&=18 \\
    |12|&=3 \\
    |13|&=18 \\
    |14|&=9 \\
    |15|&=6 \\
    |16|&=9 \\
    |17|&=18 \\
    \end{align*}
    \pagebreak
    \setitemnumber{23}
    \item Let $a,b \in G$. Prove the following statements.
    \begin{enumerate}
        \item The order of $a$ is the same as the order of $a^{-1}$.
        \begin{proof} Let $a^n = e$, then 
            \begin{align*}
                e &= {(aa^{-1})}^n \\
                  &= a^n{(a^{-1})}^n \\
                  &= e{(a^{-1})}^n \\
                  &= {(a^{-1})}^n \\
            \end{align*}
            So $|a^{-1}| \leq n$. Now we let ${(a^{-1})}^m = e$, similarly we have
            \begin{align*}
                e &= {(aa^{-1})}^m \\
                  &= a^m{(a^{-1})}^m \\
                  &= a^{m}e \\
                  &= a^m
            \end{align*}
            So $|a| \leq m$. Thus we have both $|a^{-1}| \leq n \implies m \leq n$, \\ 
            and $|a| \leq m \implies n \leq m$. Therefore $m = n$ and $|a| = |a^{-1}|$
        \end{proof}
        \item For all $g \in G, |a| = |g^{-1}ag|$
        \begin{proof} Let $|a|=n$, then $a^n=e$. Furthermore
            \begin{align*}
                {(g^{-1}ag)}^{n} &= g^{-1}agg^{-1}ag \dots g^{-1}ag \\
                                 &= g^{-1}ag && \text{(Cancelling $g^{-1}g$)} \\
                                 &= g^{-1}eg \\
                                 &= g^{-1}g \\
                                 &= e
            \end{align*}
            So $|g^{-1}ag| \leq n$. Now let $c = g$. The same steps can be used to show that 
            $|cg^{-1}agc^{-1}| \leq |g^{-1}ag|$. But $cg^{-1}agc^{-1} = gg^{-1}agg^{-1} = a$. Thus $|a| \leq |g^{-1}ag|$ or 
            $n \leq |g^{-1}ag| \leq n$. Therefore $|g^{-1}ag| = n = |a|$.
        \end{proof}
        \item The order of $ab$ is the same as the order of $ba$.
        \begin{proof} Let $|ab| = n$, then ${(ab)}^{n} = e$. We show
            \begin{align*}
                {(ba)}^n &= {(ba)}^{n}e \\
                         &= {(ba)}^{n}bb^{-1} \\
                         &= bababa \dots babb^{-1} \\
                         &= b{(ab)}^{n}b^{-1} && \text{(Associative Property)} \\
                         &= beb^{-1} \\
                         &= bb^{-1} \\
                         &= e
            \end{align*}
            So $|ba| \leq n$. Next we let $|ba| = m$, then ${(ba)}^m = e$. We can show again that
            \begin{align*}
                {(ab)}^m &= {(ab)}^{m}e \\
                         &= {(ab)}^{m}aa^{-1} \\
                         &= ababab \dots abaa^{-1} \\
                         &= a{(ba)}^{m}a^{-1} && \text{(Associative Property)} \\
                         &= aea^{-1} \\
                         &= aa^{-1} \\
                         &= e
            \end{align*}
            So $|ab| \leq m$. From here we know $m \leq n$ and $n \leq m$. \\
            Thus $n=m$ and $|ab| = |ba|$.
        \end{proof}
    \end{enumerate}
\end{enumerate}
\end{document}