\documentclass[12pt]{article}

\usepackage{geometry}
\usepackage{fancyhdr}
\usepackage{extramarks}
\usepackage{amsmath}
\usepackage{amsthm}
\usepackage{amsfonts}
\usepackage{amssymb}
\usepackage{tikz}
\usepackage[plain]{algorithm}
\usepackage{algpseudocode}
\usepackage{fancyhdr}
\usepackage{array}
\usepackage{wrapfig}
\usepackage{adjustbox}
\usepackage{enumitem}


\newlength{\strutheight}
\settoheight{\strutheight}{\strut}
\newtheorem{theorem}{Theorem}
\renewcommand\qedsymbol{$\blacksquare$}
\newcommand\setitemnumber[1]{\setcounter{enumi}{\numexpr#1-1\relax}}

\geometry{letterpaper,portrait,margin=1in}


\title{\large Abstract Algebra Homework 5}
\author{\large Zachary Meyner}
\date{}

\begin{document}
\maketitle
\begin{enumerate}[label=\textbf{\arabic*}.]
	\setitemnumber{2}
	\item Prove that $\mathbb{Z} \cong n\mathbb{Z}$ for $n \neq 0$.
	      \begin{proof} Let $\varphi: \mathbb{Z}\rightarrow n\mathbb{Z}$ with $n \neq 0$, and $\varphi(x) = nx$. \\
		      One-to-one: \\
		      Let $a,b \in \mathbb{Z}$ with $\varphi(a) = \varphi(b)$. Then
		      \begin{align*}
			      na            & = nb \\
			      \Rightarrow a & = b
		      \end{align*}
		      so $\varphi$ is one-to-one. \\
		      Onto: \\
		      Let $x \in n\mathbb{Z}$ and consider
		      \begin{align*}
			      y          & = \frac{x}{n}  \\
			      \varphi(y) & = n\frac{x}{n} \\
			      \varphi(y) & = x
		      \end{align*}
		      so $y \in \mathbb{Z}$ because $x \in n\mathbb{Z}$, and $\varphi$ is onto. \\
		      Preserves Operation: \\
		      Let $a,b \in \mathbb{Z}$, then
		      \begin{align*}
			      \varphi(a+b) & = n(a+b)                  \\
			                   & = na+nb                   \\
			                   & = \varphi(a) + \varphi(b)
		      \end{align*}
		      so $\varphi$ preserves operation.
	      \end{proof}
	      \setitemnumber{26}
	\item Let $\varphi: G \rightarrow H$ be a group isomorphism. Show that $\varphi(x) = e_H$ iff $x = e_G$,
	      where $e_G$ and $e_H$ are the identities of $G$ and $H$, respectively.
	      \begin{proof}$(\Rightarrow)$ Let $x \in G$ s.t. $\varphi(x) = e_H$. Then
            \begin{align*}
                \varphi(x)\varphi^{-1}(x) = e_H{(e_{H})}^{-1} = e_H \\
                \varphi(xx^{-1}) = \varphi(e_G) = e_H \\
                \varphi(x) = e_H
            \end{align*}
            thus $x = e_G$. \\
            $(\Leftarrow)$ Let $x = e_G$ and $h = \varphi(g) \in H$, then
            \begin{align*}
                h\varphi(x) &= \varphi(g)\varphi(x) \\
                            &= \varphi(gx) \\
                            &= \varphi(ge_G) \\
                            &= \varphi(g) \\
                            &= h \\
                            &= \varphi(e_Gg) \\
                            &= \varphi(x)\varphi(g) \\
                            &= \varphi(x)h
            \end{align*}
            So by definition of identity $\varphi(x) = e_H$.
	      \end{proof}
\end{enumerate}

\end{document}