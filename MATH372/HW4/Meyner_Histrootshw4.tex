\documentclass[12pt]{article}

\usepackage{geometry}
\usepackage{fancyhdr}
\usepackage{extramarks}
\usepackage{amsmath}
\usepackage{amsthm}
\usepackage{amsfonts}
\usepackage{amssymb}
\usepackage{tikz}
\usepackage[plain]{algorithm}
\usepackage{algpseudocode}
\usepackage{fancyhdr}
\usepackage{array}
\usepackage{wrapfig}
\usepackage{adjustbox}
\usepackage{enumitem}


\newlength{\strutheight}
\settoheight{\strutheight}{\strut}
\newtheorem{theorem}{Theorem}
\renewcommand\qedsymbol{$\blacksquare$}
\newcommand\setitemnumber[1]{\setcounter{enumi}{\numexpr#1-- -1\relax}}

\geometry{letterpaper,portrait,margin=1in}


\title{\large Historical Roots of Mathematics Homework 4}
\author{\large Zachary Meyner}
\date{}

\begin{document}
\maketitle
\begin{enumerate}
    \item The Pythagorean Theorem states that, if $a$ and $b$ are the sides of a right triangle with 
    hypotenuse $c$, then $a^2+b^2=c^2$: \\
    Find $a$ if $b=1,59$ and $c=2,49$. 
    \begin{gather*}
        c_{10}=169 \\
        b_{10}=109 \\
        \sqrt{{(169)}^2-{(109)}^2}=\sqrt{16680}=2\sqrt{4170}
    \end{gather*}
    \item Use the Babylonian approximation of $\pi=3$ to calculate
    \begin{enumerate}
        \item The circumference of a circle with diameter 20.
        \begin{gather*}
            d=20 \\
            r=10 \\
            C=2\pi r \\
            C=2(3)(10)=60
        \end{gather*}
        \item The areao fa circle with diameter 20.
        \begin{gather*}
            d=20 \\
            r=10 \\
            A=\pi r^2 \\
            A=(3){(10)}^2=300
        \end{gather*}
    \end{enumerate}
    \item Solve the following quadratic equations using the Babylonian method:
    \begin{enumerate}
        \item $x^2+4x=21$
        \begin{gather*}
            x(x+4)=21 \\
            x=a-2 \\
            (a-2)(a+2)=21 \\
            a^2-4=21 \\
            a^2=25 \\
            a=5 \\
            x=3
        \end{gather*}
        \item $x^2-2x=4$
        \begin{gather*}
            x^2-2x=4 \\
            x(x-2)=4 \\
            x=a+1 \\
            (a+1)(a-1)=4 \\
            a^2-1=4 \\
            a^2=5 \\
            a=\sqrt5 \\
            x=sqrt5+1
        \end{gather*}
        \item $x^2+x=7$
        \begin{gather*}
            x(x+1)=7 \\
            x=a-\frac{1}{2} \\
            (a-\frac{1}{2})(a+\frac{1}{2})=7 \\
            a^2-\frac{1}{4}=7 \\
            a^2=\frac{29}{4} \\
            a=\frac{\sqrt{29}}{2} \\
            x=\frac{\sqrt{29}-1}{2}
        \end{gather*}
    \end{enumerate}
    \item Using the Babylonian method, find two numbers whose difference is 3 and whose 
    product is 40. 
    \begin{gather*}
        x-y=3 \\
        xy=40 \\
        y=3+x \\
        x(x+3)=40 \\
        x=a-\frac{3}{2} \\
        (a-\frac{3}{2})(a+\frac{3}{2})=40 \\
        a^2-\frac{9}{4}=40 \\
        a^2=\frac{169}{4} \\
        a=\frac{13}{2} \\
        x=5 \quad y=8
    \end{gather*}
    \item That system we set up in class, solve
    \begin{equation*}    
        \begin{cases}
            \displaystyle\frac{y_2}{2}{(x+30)}-\displaystyle\frac{xy_1}{2}=420 \\
            y_1-y_2=20 \\
            \displaystyle\frac{y_1}{x} = \displaystyle\frac{y_1+y_2}{30}
        \end{cases}    
        \end{equation*}
    \begin{gather*}
        y_2=y_1-20, \text{ then} \\
        \frac{y_1}{x}=\frac{y_1+y_2}{30}=\frac{2y_1-20}{30}=\frac{y_1-10}{15}, \text{ then} \\
        \frac{y_2(x+30)}{2}-\frac{xy_1}{2}=\frac{(y_1-20)(x-30)}{2}-\frac{xy_1}{2}=\frac{1}{2}{(y_1x+30y_1-20x-600-y_1x)}=420 \\
        \implies 15y_1-10x-300=420 \implies 5(3y_1-2x-60)=420 \implies 3y_1-2x-60=90 \\
        \implies x=\frac{3y_1-144}{2}, \text{ finally} \\
        \frac{y_1}{x}=\frac{y_1-10}{15} \implies \frac{2y_1}{3y_1-144}=\frac{y_1-10}{15} \implies 2y_1=\frac{3y_1^2-174y_1+1440}{15} \\
        \implies 30y_1=3y_1^2-174y_1+1440 \implies 3y_1^2-204y_1+1440=0 \\
        y_1=\frac{204+\sqrt{204^2-4(3)(1440)}}{6} \implies y_1=60\\
        y_2=60-20=40 \\
        x=\frac{3(60)-144}{2}=18
    \end{gather*}
    Thus, $x=18, y_1=60, y_2=40$
\end{enumerate}

\end{document}