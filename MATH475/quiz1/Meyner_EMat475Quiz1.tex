\documentclass[12pt]{article}

\usepackage{geometry}
\usepackage{fancyhdr}
\usepackage{extramarks}
\usepackage{amsmath}
\usepackage{amsthm}
\usepackage{amsfonts}
\usepackage{amssymb}
\usepackage{tikz}
\usepackage[plain]{algorithm}
\usepackage{algpseudocode}
\usepackage{fancyhdr}
\usepackage{array}
\usepackage{wrapfig}
\usepackage{adjustbox}
\usepackage{enumitem}


\newlength{\strutheight}
\settoheight{\strutheight}{\strut}
\newtheorem{theorem}{Theorem}
\renewcommand\qedsymbol{$\blacksquare$}
\newcommand\setitemnumber[1]{\setcounter{enumi}{\numexpr#1-- -1\relax}}

\geometry{letterpaper,portrait,margin=1in}


\title{\large Linear Algebra II Quiz 1}
\author{\large Zachary Meyner}
\date{}

\begin{document}
\maketitle

\begin{enumerate}[label=\textbf{1.2.13.}]
    \item Find the general solution for the system whose augmented matrix is:
    \[
    \begin{bmatrix}
        1 & -3 & 0 & -1 & 0 & -2 \\
        0 & 1 & 0 & 0 & -4 & 1 \\
        0 & 0 & 0 & 1 & 9 & -4 \\
        0 & 0 & 0 & 0 & 0 & 0
    \end{bmatrix}
    \]
    \begin{align*}
        R_1=R_1 + 3R_2 &\Rightarrow 
        \begin{bmatrix}
            1 & 0 & 0 & -1 & -12 & 1 \\
            0 & 1 & 0 & 0 & -4 & 1 \\
            0 & 0 & 0 & 1 & 9 & -4 \\
            0 & 0 & 0 & 0 & 0 & 0
        \end{bmatrix} \\
        R_1=R_1 + R_3 &\Rightarrow 
        \begin{bmatrix}
            1 & 0 & 0 & 0 & -3 & -3 \\
            0 & 1 & 0 & 0 & -4 & 1 \\
            0 & 0 & 0 & 1 & 9 & -4 \\
            0 & 0 & 0 & 0 & 0 & 0
        \end{bmatrix} \\
    \end{align*}
    Thus:
    \[
    \begin{cases}
        x_1 = 3x_5 - 3 \\
        x_2= 1 + 4x_5 \\
        x_3 \text{ is free} \\
        x_4 = -4x_5 - 9 \\
        x_5 \text{ is free}
    \end{cases}
    \]
\end{enumerate}
\begin{enumerate}[label=\textbf{1.2.19.}]
    \item \begin{enumerate}[label=\alph*.]
        \item Conssitent with a unique solution.
        \item Inconsistent.
    \end{enumerate}
\end{enumerate}
\begin{enumerate}[label=\textbf{1.2.24}]
    \item \begin{align*}
    \begin{bmatrix}
        1 & 3 & 2 \\
        3 & h & k
    \end{bmatrix} &\rightarrow
    \begin{bmatrix}
        1 & 3 & 2 \\
        0 & h-9 & k-6
    \end{bmatrix}
\end{align*}
\begin{enumerate}[label=\alph*.]
    \item There will be no solution when $h=9$ and $k \neq 6$.
    \item There will be a unique solution when $h \neq 9$
    \item There will be infinitely many solutions when $h=9$ and $k=6$.
\end{enumerate}
\end{enumerate}
\end{document}