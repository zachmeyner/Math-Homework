\documentclass[12pt]{article}

\usepackage{geometry}
\usepackage{fancyhdr}
\usepackage{extramarks}
\usepackage{amsmath}
\usepackage{amsthm}
\usepackage{amsfonts}
\usepackage{amssymb}
\usepackage{tikz}
\usepackage[plain]{algorithm}
\usepackage{algpseudocode}
\usepackage{fancyhdr}
\usepackage{array}
\usepackage{wrapfig}
\usepackage{adjustbox}
\usepackage{enumitem}


\newlength{\strutheight}
\settoheight{\strutheight}{\strut}
\newtheorem{theorem}{Theorem}
\renewcommand\qedsymbol{$\blacksquare$}
\newcommand\setitemnumber[1]{\setcounter{enumi}{\numexpr#1-- -1\relax}}

\geometry{letterpaper,portrait,margin=1in}


\title{\large Linear Algebra II Quiz 2}
\author{\large Zachary Meyner}
\date{}

\begin{document}
\maketitle
\begin{enumerate}[label=\textbf{\arabic*.}]
    \setitemnumber{18}
    \item Do the column of $B$ span $\mathbb{R}^4$? Does the equation $B\mathbf{x}=\mathbf{y}$ have a solution for each $\mathbf{y}$ 
    in $\mathbb{R}^4$?
    \[
        B = \begin{bmatrix}
            1 & 3 & -2 & 2 \\
            0 & 1 & 1 & -5 \\
            1 & 2 & -3 & 7 \\
            -2 & -8 & 2 & -1
        \end{bmatrix} \rightarrow \ \begin{bmatrix}
            1 & 4 & -1 & \frac{1}{2} \\
            0 & 1 & 1 & \frac{-1}{13} \\
            0 & 0 & 0 & 1 \\
            0 & 0 & 0 & 0
        \end{bmatrix}
    \]
    There are only 3 pivots in $B$ so it does not span $\mathbb{R}^4$, thus the equation $B\mathbf{x}=\mathbf{y}$ does 
    not have a solution for each $\mathbf{y}$ in $\mathbb{R}^4$.
    \setitemnumber{43}
    \item Suupse $A$ is a $4 \times 3$ matrix and $\mathbf{b}$ is a vector in $\mathbb{R}^4$ with the property that $A\mathbf{x}=\mathbf{b}$ has 
    a unique solution. What can you say about the reduced echelon form of $A$? Justify 
    your answer. \\
    Since $A\mathbf{x} = \mathbf{b}$ has a unique solution we know that the reduced echelon form of $A\mathbf{x}=\mathbf{b}$ 
    augmented matrix is
    \[
        \begin{bmatrix}
            1 & 0 & 0 & a \\
            0 & 1 & 0 & b \\
            0 & 0 & 1 & c \\
            0 & 0 & 0 & 0
        \end{bmatrix}
    \]
    where $\mathbf{b} = \begin{bmatrix}
        a \\ b \\ c \\ d
    \end{bmatrix}
    $. Thus we know that the reduced row echelon form of $A$ is 
    \[
      \begin{bmatrix}
        1 & 0 & 0 \\
        0 & 1 & 0 \\
        0 & 0 & 1
      \end{bmatrix}  
    \]
    \setitemnumber{49}
    \item Determine if the columns of the matrix span $\mathbb{R}^4$.
    \[
      \begin{bmatrix}
        12 & 11 & -6 & -7 & 13 \\
        -9 & 4 & -8 & 7 & -3 \\
        -6 & 11 & -7 & 3 & -9 \\
        4 & -6 & 10 & -5 & 12
      \end{bmatrix} 
      \rightarrow \begin{bmatrix}
        1 & \frac{-7}{12} & \frac{11}{12} & \frac{-3}{4} & \frac{5}{12} \\
        0 & 1 & \frac{-1}{5} & \frac{-1}{5} & \frac{-13}{15} \\
        0 & 0 & 1 & \frac{-41}{84} & \frac{23}{18} \\
        0 & 0 & 0 & 0 & 1
      \end{bmatrix}
    \]
    The echelon form of the matrix has 4 pivot positions, thus the matrix span $\mathbb{R}^4$.
    \setitemnumber{51}
    \item Find a column of the matrix in exercise 49 that can be deleted and yet have the 
    remaining matrix columns still span $\mathbb{R}^4$. \\
    Further reducing the matrix into reduced echelon form gives 
    \[
      \begin{bmatrix}
        1 & 0 & 0 & \frac{-10}{21}& 0 \\
        0 & 1 & 0 & \frac{25}{84} & 0\\
        0 & 0 & 1 & \frac{41}{84} & 0\\
        0 & 0 & 0 & 0 &  1
      \end{bmatrix}  
    \]
    Since the fourth vector does not give a pivot row it is the only vector in the list that 
    cannot be made with a linear combination of the other 4 vectors, thus you can 
    remove one of the first, second, third, or fifth vector and it will still span $\mathbb{R}^4$.
\end{enumerate}

\end{document}