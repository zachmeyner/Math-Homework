\documentclass[12pt]{article}

\usepackage{geometry}
\usepackage{fancyhdr}
\usepackage{extramarks}
\usepackage{amsmath}
\usepackage{amsthm}
\usepackage{amsfonts}
\usepackage{amssymb}
\usepackage{tikz}
\usepackage[plain]{algorithm}
\usepackage{algpseudocode}
\usepackage{fancyhdr}
\usepackage{array}
\usepackage{wrapfig}
\usepackage{adjustbox}
\usepackage{enumitem}


\newlength{\strutheight}
\settoheight{\strutheight}{\strut}
\newtheorem{theorem}{Theorem}
\renewcommand\qedsymbol{$\blacksquare$}
\newcommand\setitemnumber[1]{\setcounter{enumi}{\numexpr#1-1\relax}}

\geometry{letterpaper,portrait,margin=1in}


\title{\large Abstract Algebra Homework 2}
\author{\large Zachary Meyner}
\date{}

\begin{document}
\maketitle
\begin{enumerate}[label=\textbf{\arabic*}.]
	\setitemnumber{41}
	\item Prove that
	      \[G=\{a+b\sqrt{2}: a,b \in \mathbb{Q} \ \text{and} \ a \ \text{and} \ b \ \text{are not both zero}\}\]
	      is a subgroup if $\mathbb{R}^*$ under the group operation of multiplication.
	      \begin{proof} Let $n,m \in G \ s.t. \ n = a+b\sqrt{2} \ \text{and} \ m = c+d\sqrt{2}$. \\
            (WTS\@: $n \cdot m \in G$ and $n^{-1} \in G$) \\
            Multiplying $m \cdot n$ we have 
            \begin{align*}
                (a+b\sqrt{2}) \cdot (c+d\sqrt{2}) &= ac+ad\sqrt{2} + bc\sqrt{2} + bd\sqrt{2}^2 && \text{(Distributive Property)} \\
                &= ac + ad\sqrt2 +bc\sqrt2 + 2bd  \\
                &= ac + \sqrt{2}(ad+bc) + 2bd && \text{(Distributive Property)} \\
                &= (ac + 2bd) + (ad + bc)\sqrt{2} && \text{(Commutative and Associative Property)}
            \end{align*}
            and $(ac+2bd) + (ad+bc)\sqrt{2}$ is clearly an in $G$, so $n \cdot m$ must be in $G$.
            Now if we take $n^{-1}$ we get
            \begin{align*}
                \frac{1}{a+b\sqrt{2}} &= \frac{1}{a+b\sqrt{2}} \cdot \frac{(a-b\sqrt{2})}{(a-b\sqrt{2})} && \text{(Multiplying by 1)} \\
                &= \frac{a-b\sqrt{2}}{(a+b\sqrt{2})\cdot(a-b\sqrt{2})} \\
                &= \frac{a-b\sqrt{2}}{a^2-b\sqrt{2}^2} && \text{(Distributive Property)} \\
                &= \frac{a+(-b)\sqrt{2}}{a^2-2b} && \text{(Simplifying)} \\
                &= \frac{a}{a^2-2b} + \frac{-b}{a^2-2b}\sqrt{2} && \text{(Commutative and Associative Propety)} \\
            \end{align*} 
            Because $a$ and $b$ are in $\mathbb{Q}$ we know $\frac{a}{a^2-2b}$ and $\frac{-b}{a^2-2b}$ must also be in $\mathbb{Q}$, so $n^{-1}$ must be in $G$. \\
            $\therefore$ by the 2 step test $G$ is a subgroup of $\mathbb{R}^*$ under the operation of multiplication.
	      \end{proof}
          \pagebreak
    \setitemnumber{45}
    \item Prove that the intersection of two subgroups of a group $G$ is also a subgroup of $G$.
          \begin{proof} Let $P \leq G$ and $H \leq G$. We know that at least the identity element $e \in P \cap H$ Let $a,b \in P \cap H$
            (WTS\@: $ab^{-1} \in P \cap H$). Because $a,b \in P \cap H$ we know
            \begin{align*}
                a &\in P \cap H \\
                \implies a &\in P \ \text{and} \ a \in H \\
                b &\in P \cap H \\
                \implies b &\in P \ \text{and} \ b \in H
            \end{align*}
            Since $P$ and $H$ are subgroups of G we have
            \begin{align*}
                &ab^{-1} \in P \ \text{and} \ ab^{-1} \in H \\
                \implies &ab^{-1} \in P \cap H
            \end{align*}
            Thus $P \cap H$ is also a subgroup of $G$.
          \end{proof}
\end{enumerate}
\end{document}