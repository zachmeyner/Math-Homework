\documentclass[12pt]{article}

\usepackage{geometry}
\usepackage{fancyhdr}
\usepackage{extramarks}
\usepackage{amsmath}
\usepackage{amsthm}
\usepackage{amsfonts}
\usepackage{amssymb}
\usepackage{tikz}
\usepackage[plain]{algorithm}
\usepackage{algpseudocode}
\usepackage{fancyhdr}
\usepackage{array}
\usepackage{wrapfig}
\usepackage{adjustbox}
\usepackage{enumitem}
\usepackage{tasks}


\newlength{\strutheight}
\settoheight{\strutheight}{\strut}
\newtheorem{theorem}{Theorem}
\renewcommand\qedsymbol{$\blacksquare$}
\newcommand\setitemnumber[1]{\setcounter{enumi}{\numexpr#1-- -1\relax}}

\geometry{letterpaper,portrait,margin=1in}


\title{\large Historical Roots of Mathematics Homework 1}
\author{\large Zachary Meyner}
\date{}

\begin{document}
\maketitle
\begin{enumerate}[label=\arabic*.]
    \setitemnumber{1}
    \item Write the follwing problems in hieroglyphics and then perform the addition:
    \begin{enumerate}[label=\alph*.]
        \item $46+23$ \vspace{50mm}
        \item $64+28$ \vspace{50mm}
        \item $4297+1351$ \vspace{50mm}
    \end{enumerate}
    \item 
    \begin{enumerate}[label=\alph*.]
        \item Show that 
        \[ \frac{2}{n} = \frac{1}{3n}+\frac{5}{3n} \]
        hence that $2/n$ can be expressed as a sum of unit fractions whenever $n$ is divisible by 5. \\
        That is 
        If $n=5k \ k \in \mathbb{Z}$, then $\displaystyle\frac{2}{n} = \displaystyle\frac{1}{3n} + \displaystyle\frac{1}{3k}$.
        \begin{proof}
            Let $n,k \in \mathbb{Z}$ s.t. $n = 5k$. Then
            \begin{align*}
                \frac{2}{n} &= \frac{6}{3n} \\
                &= \frac{1}{3n} + \frac{5}{3n} \\
                &= \frac{1}{3n} + \frac{5}{3(5k)} \\
                &= \frac{1}{3n} + \frac{1}{3k}
            \end{align*}
        \end{proof}
        \item Note: $25 = 5(5)$
        \[\frac{2}{25} = \frac{1}{75} + \frac{1}{15} \]
        Note: $65 = 5(13)$
        \[\frac{2}{65} = \frac{1}{195} + \frac{1}{39} \]
        Note: $85 = 5(17)$
        \[\frac{2}{85} = \frac{1}{255} + \frac{1}{51} \] 
    \end{enumerate}
    \item Use the $2 \div n$ table to write the following as sums of unit fractions without repetition:
    \begin{enumerate}[label=\alph*.]
        \item \begin{align*}
            \frac{13}{15} &= \frac{3}{15} + \frac{2}{3} \\ 
            &= \frac{1}{5} + \frac{1}{2} + \frac{1}{6} \\
        \end{align*}
        \item \begin{align*}
            \frac{9}{49} &= \frac{7}{49} + \frac{2}{49} \\
            &= \frac{1}{7} + \frac{1}{29} + \frac{1}{196} \\
        \end{align*}
        \item \begin{align*}
            \frac{19}{35} &= \frac{5}{35} + \frac{2}{5} \\
            &= \frac{1}{7} + \frac{1}{3} + \frac{1}{15}
        \end{align*}
    \end{enumerate} 
    \item Compute in the Anceint Egyptian way:
    \begin{tasks}[label=\alph*.] (2)
        \task$3 \div 4$
        \begin{align*}
            &1 & 4 \\
            \checkmark&\overline{2} & 2 \\
            \checkmark&\overline{4} &1\\
        \end{align*}
        $3\div 4 = \overline{2} + \overline{4}$
        \task$5 \div 8$
        \begin{align*}
            &1 & 8 \\
            \checkmark&\overline{2} & 4 \\
            &\overline{4} & 2 \\
            \checkmark&\overline{8} & 1
        \end{align*}
        $5 \div 8=\overline{2} + \overline{8}$
        \task$14 \div 24$
        \begin{align*}
            &1 & 24 \\
            \checkmark&\overline{3} & 8\\
            \checkmark&\overline{2} & 12\\
            \checkmark&\overline{4} & 6
        \end{align*}
        $14 \div 24 = \overline{3} + \overline{4}$
        \task$35 \div 32$
        \begin{align*}
            \checkmark&1 & 32 \\
            &\overline{2} & 16\\
            &\overline{4} & 8 \\
            &\overline{8} & 4 \\
            \checkmark&\overline{16} & 2 \\
            \checkmark&\overline{32} & 1
        \end{align*}
        $35 \div 32 = 1 + \overline{16} + \overline{32}$
        \task$5 \div 6$
        \begin{align*}
            &1 & 6 \\
            \checkmark&\overline{\overline{3}} & 4\\
            \checkmark&\overline{6} & 1 
        \end{align*}
        $5 \div 6 = \overline{\overline{3}} + \overline{6}$ 
        \task$17 \div 12$
        \begin{align*}
            \checkmark&1 & 12\\
            \checkmark&\overline{3} & 4\\
            \checkmark&\overline{12} &1
        \end{align*}
        $17 \div 12 = 1 + \overline{3} + \overline{12}$
        \task$11 \div 16$
        \begin{align*}
            &1 & 16 \\
            \checkmark&\overline{2} & 8 \\
            &\overline{4} & 4 \\
            \checkmark&\overline{8} & 2 \\
            \checkmark&\overline{16} & 1
        \end{align*}
        $11 \div 16 = \overline{2} + \overline{8} + \overline{16}$
        \task$51 \div 18$
        \begin{align*}
            &1 & 18 \\
            \checkmark&2 & 36 \\
            \checkmark&\overline{\overline{3}} & 12 \\
            &\overline{3} & 6 \\
            \checkmark&\overline{6} & 3
        \end{align*}
        $51 \div 18 = 2 + \overline{\overline{3}} + \overline{6}$
    \end{tasks}
    \item Find a Sylveester-type representation (as a sum of unit fractions) for each of the following:
    \begin{tasks}[label=\alph*.] (2)
        \task\begin{align*}
            \frac{13}{36} &= \frac{1}{36} + \frac{1}{3}
        \end{align*}
        \task\begin{align*}
            \frac{9}{20} &= \frac{4}{20} + \frac{1}{4} \\
            &= \frac{1}{5} + \frac{1}{4}
        \end{align*}
        \task\begin{align*}
            \frac{4}{15} &= \frac{1}{15} + \frac{1}{5}
        \end{align*}
        \task\begin{align*}
            \frac{335}{336} &= \frac{167}{336} + \frac{1}{2} \\
            &= \frac{55}{336} + \frac{1}{2} + \frac{1}{3} \\
            &= \frac{7}{336} + \frac{1}{2} + \frac{1}{3} + \frac{1}{7} \\
            &= \frac{1}{48} + \frac{1}{2} + \frac{1}{3} + \frac{1}{7} 
        \end{align*}
    \end{tasks} 
\end{enumerate}

\end{document}