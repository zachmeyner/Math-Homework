\documentclass[11pt]{article}

\usepackage{geometry}
\usepackage{fancyhdr}
\usepackage{extramarks}
\usepackage{amsmath}
\usepackage{amsthm}
\usepackage{amsfonts}
\usepackage{amssymb}
\usepackage{tikz}
\usepackage[plain]{algorithm}
\usepackage{algpseudocode}
\usepackage{fancyhdr}
\usepackage{array}
\usepackage{wrapfig}
\usepackage{adjustbox}
\usepackage{enumitem}
\usepackage{fontspec,unicode-math}



\newlength{\strutheight}
\settoheight{\strutheight}{\strut}
\newtheorem{theorem}{Theorem}
\renewcommand\qedsymbol{$\blacksquare$}
\newcommand\setitemnumber[1]{\setcounter{enumi}{\numexpr#1-- -1\relax}}

\newenvironment{rcases}
                {\left.\begin{aligned}}
                {\end{aligned}\right\rbrace}

\geometry{letterpaper,portrait,margin=1in}


\title{\large Historical Roots of Mathematics Homework 6}
\author{\large Zachary Meyner}
\date{}

\setmainfont{STIXTwoText}[
    Extension =.otf,
    UprightFont =*-Regular,
    ItalicFont =*-Italic,
    BoldFont =*-Bold,
    BoldItalicFont = *-BoldItalic
]
\setmathfont{STIXTwoMath-Regular.otf}

\begin{document}
\maketitle

\begin{enumerate}
    \item It is said that Pythagoras himself genereated Pythagorean triples using the following method:\\
    Let $n \in \mathbb{Z}^{+}$ and let 
        
    \begin{align*}
        \begin{rcases}
            x = 2n+1 \\
            y = 2n^2+2n \\
            z=2n^2+2n+1
        \end{rcases}
        &&\text{(*)}
    \end{align*}
    \begin{enumerate}
        \item Find $x$, $y$, and $z$ for
        \begin{enumerate}
            \item $n=1$
            \begin{gather*}
                x=2(1)+1=3 \\
                y=2{(1)}^2+2(1)=4 \\
                z=2{(1)}^2+2(1)+1=5
            \end{gather*}
            \item $n=2$
            \begin{gather*}
                x=2(2)+1=5 \\
                y=2{(2)}^2+2(2)=12 \\
                z=2{(2)}^2+2(2)+1=13
            \end{gather*}
            \item $n=3$
            \begin{gather*}
                x=2(1)+1=7 \\
                y=2{(1)}^2+2(1)=24 \\
                z=2{(1)}^2+2(1)+1=25
            \end{gather*}
        \end{enumerate}
        \item Prove that (*) produces a Pythagorean triple for any positive integer $n$.
        \begin{proof}
            Let $x,y,z \in \mathbb{Z}^{+}$ s.t.
            \[
            \begin{rcases}
                x=2n+1 \\
                y=2n^2+2n \\
                z=2n^2+2n+1
            \end{rcases}
            \]
            with $n \in \mathbb{Z}^{+}$. Then:
            \begin{align*}
                x^2+y^2 &= {(2n+1)}^2 + {2n^2+2n}^2 \\
                &= (4n^2+4n+1) + (4n^4+8n^3+4n^2) \\
                &= 4n^4+8n^3+8n^2+4n+1
            \end{align*}
            Also: 
            \begin{align*}
                z^2 &= {(2n^2+2n+1)}^2 \\
                &= 4n^4+8n^3+8n^2+4n+1
            \end{align*}
            Thus $x^2+y^2=z^2$ and (*) produces Pythagorean triples.
        \end{proof}
        \item Will (*) produce every primitive Pythagorean triple? Explain. \\
        No (*) will not generate every Pythagorean triple. It only generates Pythagorean triples 
        where $y$ and $z$ differ by $1$, and there are primitive Pythagorean triples where $y$ and $z$ 
        differ by more than 1, eg $(8, 15, 17)$.
    \end{enumerate}
    \item Write each of the following numbers as the sum of three of fewer triangular numbers:
    \begin{enumerate}
        \item $56 = t_{10}+t_{1}=55+1$
        \item $69 = t_{11}+t{2}=66+3$
        \item $185 = t_{13}+t_{13}+t_{2}=91+91+3$
        \item $287=t_{22}+t_{7}+t_{3}=253+28+6$
    \end{enumerate}
    \item Use mathematical induction to prove that $t_n = \displaystyle\frac{n(n+1)}{2}$.
    \begin{proof}
        Let $t_n = \sum\limits_{i=1}^{n}i$ be the $n$th triangular number. \\
        Base case ($n=1$): \\
        Let $n=1$, then $\sum\limits_{i=1}^{1}i = 1$ and $\displaystyle\frac{1(1+1)}{2}=\frac{2}{2}=1$. \\
        Hypothesis ($n=k$): \\
        Assume $t_k= \sum\limits_{i=1}^{k}i=1+2+\cdots+k=\displaystyle\frac{k(k+1)}{2}$ $\forall k \in \mathbb{Z}^{+}$ \\
        Induction ($n=k+1$): \\
        Consider $\sum\limits_{i=1}^{k+1}i$, then
        \begin{align*}
            \sum_{i=1}^{k+1}i &= 1+2+\cdots+k+(k+1) \\
            &= \frac{k(k+1)}{2} + (k+1) && \text{(Hypothesis)} \\
            &= \frac{k^2+k+2(k+1)}{2} \\
            &= \frac{k^2+3k+2}{2} \\
            &= \frac{(k+1)(k+2)}{2} = \frac{(k+1)((k+1)+1)}{2}
        \end{align*}
        \[\therefore t_n=\sum\limits_{i=1}^{n}i=\displaystyle\frac{n(n+1)}{2}\]
    \end{proof}
    \item Prove that 1184 and 1210 are amicable.
    \begin{proof}
    Divisors of 1184 are $\{1, 2, 4, 8, 16, 32, 37, 74,148, 296, 592, 1184\}$ \\
    $1+2+4+16+32+37+74+148+296+592=1210$ \\
    Divisors of 1210 are $\{1, 2, 5, 10, 11, 22, 55, 110, 121, 242, 605, 1210\}$ \\
    $1+2+5+10+11+22+55+110+121+242+605=1184$ \\
    Thus 1184 and 1210 are amicable.
    \end{proof}
    
    \item Verify that ${\bigg(\frac{m^2-1}{2}\bigg)}^2+m^2={\bigg(\frac{m^2+1}{2}\bigg)}^2$.
    \begin{align*}
        {\Bigg(\frac{m^2-1}{2}\Bigg)}^2+m^2 &= \frac{m^4-2m^2+1}{4}+m^2 \\
        &= \frac{m^4-2m^2+1+4m^2}{4} \\
        &= \frac{m^4+2m^2+1}{4} \\
        &= {\Bigg(\frac{m^2+1}{2}\Bigg)}^2
    \end{align*}
\end{enumerate}


\end{document}