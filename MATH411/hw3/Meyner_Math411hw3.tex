\documentclass[12pt]{article}

\usepackage{geometry}
\usepackage{fancyhdr}
\usepackage{extramarks}
\usepackage{amsmath}
\usepackage{amsthm}
\usepackage{amsfonts}
\usepackage{amssymb}
\usepackage{tikz}
\usepackage[plain]{algorithm}
\usepackage{algpseudocode}
\usepackage{fancyhdr}
\usepackage{array}
\usepackage{wrapfig}
\usepackage{adjustbox}
\usepackage{enumitem}


\newlength{\strutheight}
\settoheight{\strutheight}{\strut}
\newtheorem{theorem}{Theorem}
\renewcommand\qedsymbol{$\blacksquare$}
\newcommand\setitemnumber[1]{\setcounter{enumi}{\numexpr#1-1\relax}}

\geometry{letterpaper,portrait,margin=1in}


\title{\large Intro to Analysis Homework 3}
\author{\large Zachary Meyner}
\date{}

\begin{document}
\maketitle
\begin{enumerate}
	\item Let $a_1 = 1$ and define $a_{n+1} = 1 + \frac{a_n}{2}$. Show that the sequence converges and make an
	      educated guess for the limit value. \\
	      Guess: $a_n \rightarrow 2$
	      \begin{gather*}
		      a_n \leq a_{n+1} \\
		      a_n \leq 1 + \frac{a_n}{2} \\
		      \frac{a_n}{2} \leq 1 \\
		      a_n \leq 2
	      \end{gather*}
	      \begin{proof}
		      \underline{Claim 1}: ($a_n$) is bounded above by 2. \\
		      We prove this using induction. \\
		      Base case: ($n = 2$)
		      \begin{align*}
			      a_2 & = 1+\frac{a_1}{2} \\
			          & = 1+\frac{1}{2}   \\
			          & = \frac{3}{2}
		      \end{align*}
		      Inductive Hypothesis: ($n = k$) \\
		      Assume that $a_k < 2$ \\
		      Inductive Step ($n = k+1$)
		      \begin{align*}
			      a_{k+1} \leq 2  \Rightarrow 1+\frac{a_k}{2} & \leq 2                                    \\
			      \Rightarrow \frac{a_k}{2}                   & \leq 1                                    \\
			      \Rightarrow a_k                             & \leq 2 &  & \text{(Inductive Hypothesis)} \\
		      \end{align*}
		      $\therefore a_n$ is bounded above by $2$. \\
		      \underline{Claim 2}: ($a_n$) is increasing. \\
		      By claim 1,
		      \begin{align*}
			      a_n                       & \leq 2               \\
			      \Rightarrow \frac{a_n}{2} & \leq 1               \\
			      \Rightarrow a_n           & \leq 1+\frac{a_n}{2} \\
			      \Rightarrow a_n           & \leq a_{n+1}
		      \end{align*}
		      Thus ($a_n$) is decreasing, and by the Monotone Convergence Theorem it converges.
	      \end{proof}
	\item Let $(x_n) = \frac{2^n}{n!}$. Show that this sequence converges using the Monotone convergence
	      theorem. \\
	      Guess: $(x_n) \rightarrow 0$
	      \begin{proof}
		      \underline{Clain 1}: ($x_n$) is bounded below by 0. \\
		      This is clearly true because $2^n > 0$ and $n! > 0$. \\
		      \underline{Claim 2}: ($x_n$) is decreasing.
		      \begin{align*}
			      \frac{2^n}{k!} & \geq \frac{2^{n+1}}{(n+1)!} \\
			      2^n(n+1)!      & \geq 2^{n+1}n!              \\
			      2^n(n+1)n!     & \geq 2^n2n!                 \\
			      n+1            & \geq 2                      \\
			      n              & \geq 1
		      \end{align*}
		      and $n \in \mathbb{N}$ so this is clearly true. Thus ($x_n$) is decreasing. \\
		      By the Monotone Convergence Theorem ($x_n$) converges.
	      \end{proof}
	\item Let $x_1 = 1$ and $x_{n+1} = \frac{4+3x_n}{3+2x_2}$. Show that this sequence converges by the monotone
	      covergence theorem. \\
	      Guess: $x_n \rightarrow \sqrt{2}$.
	      \begin{gather*}
		      x_n < x_{n+1} \\
		      x_n < \frac{4+3x_n}{3+2x_n} \\
		      x_n(3+2x_n) < 4+3x_n \\
		      2x_n^2+3x_n < 4+3x_n \\
		      2x_n^2 < 4 \\
		      x_n^2 < 2 \\
		      x_n < \sqrt{2}
	      \end{gather*}
	      \begin{proof}
		      \underline{Claim 1}: ($x_n$) is bounded above by $\sqrt{2}$. \\
		      Proof by induction: \\
		      Base case: ($n=2$)
		      \begin{align*}
			      x_2 & = \frac{4+3(1)}{3+2(1)}  \\
			          & = \frac{7}{5} < \sqrt{2}
		      \end{align*}
		      Inductive Hypothesis ($n=k$) \\
		      Assume $x_k \leq \sqrt2$ \\
		      Inductive Step ($n=k+1$)
		      \begin{align*}
			      x_{k+1} \leq \sqrt2 \Rightarrow \frac{4+3x_n}{3+2x_n} & \leq \sqrt2                                                               \\
			      \Rightarrow 4+3x_n                                    & \leq \sqrt2(3+2x_n)                                                       \\
			      \Rightarrow 4+3x_n                                    & \leq 3\sqrt2+2x_n\sqrt2                                                   \\
			      \Rightarrow 3x_n - 2x_n\sqrt2                         & \leq 3\sqrt2 - 4                                                          \\
			      \Rightarrow x_n(3-2\sqrt2)                            & \leq 3\sqrt2-4                                                            \\
			      \Rightarrow x_n                                       & \leq \frac{3\sqrt2-4}{3-2\sqrt2} \bigg(\frac{3+2\sqrt2}{3+2\sqrt2} \bigg) \\
			      \Rightarrow x_n                                       & \leq \frac{9\sqrt2+12-12-8\sqrt2}{9-8}                                    \\
			      \Rightarrow x_n                                       & \leq \frac{\sqrt2}{1}                                                     \\
			      \Rightarrow x_n                                       & \leq \sqrt2
		      \end{align*}
			  $\therefore x_n$ is bounded above by $\sqrt2$. \\
			  \underline{Claim 2}: ($x_n$) is decreasing. \\
			  By clain 1,
			  \begin{align*}
				  x_n \leq \sqrt2 \Rightarrow x_n^2 &\leq 2 \\
				  2x_n^2 &\leq 4 \\
				  2x_n^2 + 3x_n &\leq 4 + 3x_n \\
				  x_n(2x_n + 3) &\leq 4+3x_n \\
				  x_n &\leq \frac{4+3x_n}{3+2x_n} \\
				  x_n &\leq x_{n+1}
			  \end{align*}
			  Thus ($x_n$) is decreasing. \\
			  By the Monotone Convergence Theorem ($x_n$) converges.
	      \end{proof}
	\item Establish the divergence of the folowing sequences. You can use any of the divernce 
	criteria in 3.4.
	\begin{enumerate}
		\item $\frac{2n^2(-1+^n+3n)}{n^2}$ \\
		Consider the subsequence 
		\begin{align*}
			x_{2n} &= \frac{2{(2n)}^2{(-1)}^{2n}+3{(2n)}}{2{(2n)}^2} \\
			&= \frac{2(4n^2)+6n}{4n^2} \\
			&= \frac{8n^2+ 6n}{4n^2} \\
			\Rightarrow \lim_{n \to \infty} x_{2n} &= 2
		\end{align*}
		Now also consider the subsequence
		\begin{align*}
			x_{2n+1} &= \frac{2{(2n+1)}^2(-1)+3{(2n+1)}}{{(2n+1)}^2} \\
			&= \frac{-2{(4n^2+4n+1)}+6n+3}{4n^2+4n+1} \\
			&= \frac{-8n^2-8n-2+6n+3}{4n^2+4n+1} \\
			&= \frac{-8n^2-2n+1}{4n^2+4n+1} \\
			\Rightarrow \lim_{n \to \infty}x_{2n+1} &= -2
		\end{align*}
		Since $x_{2n} \rightarrow 2$ and $x_{2n+1} \rightarrow -2, (x_n)$ does not converge.
		\item ${(-n)}^3$ 
		\begin{align*}
			{(-n)}^3 &= {(-1)}^3n^3 \\
			&= (-1)n^3
		\end{align*}
		Which goes to $-\infty$, so it is not bounded below and diverges.
	\end{enumerate}
	\item Suppose ($x_n$) is a bounded subsequence with $s = \sup(x_n:n \in \mathbb{N})$ and suppose further 
	that $s \notin {(x_n)}$. Construct a subsequnce which converges to $s$. \\
	Consider $s-\frac{1}{k} \ \forall k \in \mathbb{N}$. We know that because $s-\frac{1}{k} < s$ so $s-\frac{1}{k}$ cannot be an upper 
	bound on ($x_n$). Therefore $\exists x_{n_k} \in (x_n)$ where $x_{n_k} \geq s-\frac{1}{k}$. So 
	\begin{gather*}
		s \geq x_{n_k} \geq s-\frac{1}{k} \\
		\lim_{k \to \infty} s \geq \lim_{k \to \infty} x_{n_k} \geq \lim_{k \to \infty}\bigg(s-\frac{1}{k}\bigg) \\
		s \geq \lim_{k \rightarrow \infty} x_{n_k} \geq s
	\end{gather*}
	So by the Squeeze Theorem $\lim\limits_{k \to \infty} x_{n_k} = s$
	\item Show that the sequence $x_n = \frac{{(2n^2-3n)}\sin{(n^2)}}{n^2+3n+5}$ has a convergent subsequence.
		  \begin{align*}
			  |x_n| &= \frac{|2n^2-3n||\sin{(n^2)}|}{|n^2+3n+5|} \\
			  & \leq \frac{|2n^2-3n|}{|n^2+3n+5|} \\
			  & \leq \frac{|2n^2| + |-3n|}{n^2+3n+5} && \text{(Triangle Inequality)} \\
			  & \leq \frac{2n^2+3n}{n^2+3n+5} \\
			  & \leq \frac{2n^2+3n}{n^2} \\
			  & \leq \frac{6n^2}{n^2} && \text{(See Note*)} \\
			  &= 6
		  \end{align*}
		  So ($x_n$) is bounded and by the Bolzano-Weirstrass Theorem has a convergent subsequence.
		  (Note*)
		  \begin{gather*}
			  2n^2+3n \leq 8n^2 \\
			  0 \leq 6n^2-3n \\
			  0 \leq n(6n-3) \\
			  0 \leq 6n-3 \ \forall n \in \mathbb{N}
		  \end{gather*}
\end{enumerate}
\end{document}