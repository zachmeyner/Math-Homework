\documentclass[12pt]{article}

\usepackage{geometry}
\usepackage{fancyhdr}
\usepackage{extramarks}
\usepackage{amsmath}
\usepackage{amsthm}
\usepackage{amsfonts}
\usepackage{amssymb}
\usepackage{tikz}
\usepackage[plain]{algorithm}
\usepackage{algpseudocode}
\usepackage{fancyhdr}
\usepackage{array}
\usepackage{wrapfig}
\usepackage{adjustbox}
\usepackage{enumitem}
\usepackage{tasks}


\newlength{\strutheight}
\settoheight{\strutheight}{\strut}
\newtheorem{theorem}{Theorem}
\renewcommand\qedsymbol{$\blacksquare$}
\newcommand\setitemnumber[1]{\setcounter{enumi}{\numexpr#1-- -1\relax}}

\geometry{letterpaper,portrait,margin=1in}


\title{\large Hist Roots Homework 3}
\author{\large Zachary Meyner}
\date{}

\begin{document}
\maketitle

\begin{enumerate}
    \item Give three numbers that might be represented by the symbols I don't know how to type
        \begin{enumerate}
            \item 2, 120, 7200
            \item 12, 720, 2,592,000
        \end{enumerate}
    \item Find a first, second, and third approximation to:

            \begin{tasks} (2)
            \task$\sqrt{10}$\\
            One:
            \begin{gather*}
                3^2 = 9 < 10, \text{ so} \\
                3 < \sqrt{10} \\
                \frac{10}{3} > \sqrt{10}, \text{ so} \\
                4 < \sqrt{10} < \frac{10}{3}, \text{thus} \\
                \sqrt{10} \approx \frac{4+\frac{10}{3}}{2}=\frac{22}{6}=3.\overline{6}
            \end{gather*}
            Two:
            \begin{gather*}
                \frac{22}{6}\cdot\frac{60}{22}=10, \text{ so} \\
                \sqrt{10} \approx \frac{\frac{22}{6}+\frac{60}{22}}{2}=\frac{211}{66}=3.19\overline{69}
            \end{gather*}
            Three:
            \begin{gather*}
                \frac{211}{66}\cdot\frac{660}{211}=10, \text{ so} \\
                \sqrt{10} \approx \frac{\frac{211}{66}+\frac{660}{211}}{2}=3.1624\dots
            \end{gather*}
            \task$\sqrt{7}$\\
            One:
            \begin{gather*}
                2^2=4 < 7, \text{ so} \\
                2 < \sqrt{7} \\
                \frac{7}{2} > \sqrt{7}, \text{ so} \\
                2 < \sqrt{7} < \frac{7}{2}, \text{ thus} \\
                \sqrt{7} \approx \frac{2+\frac{7}{2}}{2}=11/4=2.75
            \end{gather*}
            Two:
            \begin{gather*}
                \frac{11}{4}\cdot\frac{28}{11}=7, \text{ thus} \\
                \sqrt{7} \approx \frac{\frac{11}{4}+\frac{28}{11}}{2}=\frac{233}{88}=2.647\overline{72}
            \end{gather*}
            Three: 
            \begin{gather*}
                \frac{233}{88}\cdot\frac{616}{233}=7, \text{ thus} \\
                \sqrt{7} \approx \frac{\frac{233}{88}+\frac{616}{233}}{2}=2.64575\dots
            \end{gather*}
        \end{tasks}
        \pagebreak
        \item Use the \textbf{Binomial Theorem}
        \[{(1+x)}^m=1+\sum_{k=1}^{\infty}\binom{m}{k}x^k\] 
        to prove that
        \[\sqrt{a^2+b}\approx a+ \frac{b}{2a}-\frac{b^2}{8a^2}\]
        \begin{proof}
            Consider ${(a^2+b)}^{\frac{1}{2}}$. Factoring out $a$ we have $a{(1+\frac{b}{a^2})}^{\frac{1}{2}}$. Thus we have
            \begin{align*}
                a{\bigg(1+\frac{b}{a^2}\bigg)}^\frac{1}{2}&= a\bigg(1+\sum_{k=1}^{\infty}\binom{\frac{1}{2}}{k}\bigg(\frac{b}{a^2}\bigg)^k\bigg) \\
                &\approx a\bigg(1+\frac{b}{2a^2}-\frac{b^2}{8a^3}\bigg) \\
                &=a+\frac{b}{2a}-\frac{b^2}{8a^2}
            \end{align*}
        \end{proof}
        \item Let $n=2^a3^b5^c$, where $a,b,c\in \mathbb{N}$. Prove that $n$ is sexagesimally regular.
        \begin{proof}
            Let $n=2^a3^b5^c$ s.t. $a,b,c\in\mathbb{N}$. Then we know that
            \begin{gather*}
                n|2^a \\
                n|3^b \\
                n|5^c
            \end{gather*}
            Thus we can also say,
            \begin{gather*}
                n|2^a3^b5^c
            \end{gather*}
            WLOG let $a=2m, b=c=m, \ m\in\mathbb{N}$. Then we have 
            \begin{gather*}
                n|2^{2m}3^m5^m \\
                n|{(2^2\cdot 3\cdot 5)}^m \\
                n|60^m
            \end{gather*} Thus $nz=60^m,\ z\in\mathbb{N}$, so 
            \[\frac{1}{n}=\frac{a_{1}60^m+a_{2}60^{m-1}+\cdots+a_m}{60^m}=\frac{a_1}{60}+\frac{a_2}{60^2}\cdots\frac{a_m}{60^m}\]
            $\therefore n$ is sexagesimally regular.
        \end{proof}
        \item Write in base-60:
        \begin{enumerate}
            \item $\frac{5}{6}$
            \[
              \frac{5}{6}=\frac{\frac{5}{6}(60)}{60} = \underline{\frac{50}{60}}
            \]
            \item $1\frac{4}{9}$
            \[
                \frac{4}{9} = \frac{\frac{4}{9}(60)}{60}=\frac{26}{60}+\frac{\frac{2}{3}}{60}\]
                \[\frac{26}{60}+\frac{\frac{2}{3}(60)}{60^2}=\frac{26}{60}+\frac{40}{60^2}
                \]
                \[\underline{1+\frac{26}{60}+\frac{40}{60^2}}\]
            \item $86\frac{1}{90}$
            \[
                86=1,26 \\
                \frac{1}{90}=\frac{\frac{1}{90}(60)}{60}=\frac{\frac{2}{3}}{60} \\
            \]
            \[\frac{\frac{2}{3}(60)}{60^2}=\frac{40}{60^2}\]
            \[\underline{1,26+\frac{40}{60^2}}\]
        \end{enumerate}
\end{enumerate}

\end{document}