\documentclass[12pt]{article}

\usepackage{geometry}
\usepackage{fancyhdr}
\usepackage{extramarks}
\usepackage{amsmath}
\usepackage{amsthm}
\usepackage{amsfonts}
\usepackage{amssymb}
\usepackage{tikz}
\usepackage[plain]{algorithm}
\usepackage{algpseudocode}
\usepackage{fancyhdr}
\usepackage{array}
\usepackage{wrapfig}
\usepackage{adjustbox}
\usepackage{enumitem}


\newlength{\strutheight}
\settoheight{\strutheight}{\strut}
\newtheorem{theorem}{Theorem}
\renewcommand\qedsymbol{$\blacksquare$}
\newcommand\setitemnumber[1]{\setcounter{enumi}{\numexpr#1-- -1\relax}}

\geometry{letterpaper,portrait,margin=1in}


\title{\large Abstract Algebra Homework 10}
\author{\large Zachary Meyner}
\date{}

\begin{document}
\maketitle
\begin{enumerate}[label=\textbf{\arabic*}.]
    \setitemnumber{5}
    \item Prove or disprove: If $D$ is an integral domain, then every prime element in $D$ is also 
    irreducible in $D$.
    \begin{proof}
        Let $D$ be an integral domain and let $p \in D$ be prime. Consider $p = ab$ for some 
        $a,b \in D$. Because $p$ is prime $p|a$ or $p|b$. WLOG assume $p|a$. Then $pn = a$ for some 
        $n \in D$. Then we have
        \begin{align*}
            a &= pn \\
            &= (ab)n \\
            &= a(bn)
        \end{align*}
        Because $D$ is an integral domain we can use the cancellation law and get $1=bn$. So $b$ 
        is a unit. Thus $p$ is irreducible.
    \end{proof}

    \setitemnumber{14}
    \item Let $D$ be a Euclidean domain with Euclidean valuation $\nu$. If $u$ is a unit in $D$, show 
    $\nu(u) = \nu(1)$.
    \begin{proof}
        Let $D$ be a Euclidean Domain and $u \in D$ be a unit. Since $u$ is a unit we know 
        $ua = 1 \ \forall a \in D$. Since $D$ is a Euclidean Domain we also know 
        \[\nu(u) \leq \nu(ua) = \nu(1) \leq \nu(1 \cdot u) = \nu(u)\]
        And since $\nu(u) \leq \nu(1) \leq \nu(u)$ we know that $\nu(u) = \nu(1)$. 
    \end{proof}

    \item Let $D$ be a Euclidean Domain with Euclidean valuation $\nu$. If $a$ and $b$ are associates in 
    $D$, prove that $\nu(a) = \nu(b)$.
    \begin{proof}
        Let $D$ be a Euclidean Domain with $a,b \in D$ being associates. We know $\exists u \in D$ 
        s.t. $a = bu$. Because $D$ is a Euclidean Domain 
        \[\nu(b) \leq \nu(bu) = \nu(a) \leq \nu(au^{-1}) = \nu(b)\]
        Since $\nu(b) \leq \nu(a) \leq \nu(b)$ we know that $\nu(a) = \nu(b)$. 
    \end{proof}
\end{enumerate}

\end{document}