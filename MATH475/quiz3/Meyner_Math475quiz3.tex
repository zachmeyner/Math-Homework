\documentclass[12pt]{article}

\usepackage{geometry}
\usepackage{fancyhdr}
\usepackage{extramarks}
\usepackage{amsmath}
\usepackage{amsthm}
\usepackage{amsfonts}
\usepackage{amssymb}
\usepackage{tikz}
\usepackage[plain]{algorithm}
\usepackage{algpseudocode}
\usepackage{fancyhdr}
\usepackage{array}
\usepackage{wrapfig}
\usepackage{adjustbox}
\usepackage{enumitem}


\newlength{\strutheight}
\settoheight{\strutheight}{\strut}
\newtheorem{theorem}{Theorem}
\renewcommand\qedsymbol{$\blacksquare$}
\newcommand\setitemnumber[1]{\setcounter{enumi}{\numexpr#1-- -1\relax}}

\geometry{letterpaper,portrait,margin=1in}


\title{\large Linear Algebra II Quiz 3}
\author{\large Zachary Meyner}
\date{}

\begin{document}
\maketitle
    \begin{enumerate}
        \item Assume $A$ is an $m \times n$ matrix and $CA=I_n$ the $n \times n$ identity matrix. Show that the 
        equation $Ax=0$ has only the trivial solution. Explain why the matrix $A$ can not have 
        more columns than rows.
        \begin{proof}
            Let $A$ be an $m \times n$ matrix s.t. $CA=I_n$ where $I_n$ is the $n \times n$ identity matrix. 
            Consider the equation $Ax=0$
            \begin{align*}
                Ax&=0 \\
                CAx&=C0 \\
                (CA)x&=C0 && \text{(Associative Property)} \\
                I_{n}x &= 0 \\
                x&=0
            \end{align*}
            Thus $Ax=0$ can only have the trivial solution.
        \end{proof}
        If $A$ had more columns than rows then the columns of $A$ would be be linearly dependent, 
        thus $Ax=0$ wuold have a nontrivial solution.
        \item Assume $A$ is an $m \times n$ matrix nad $AD=I_m$ the $m \times m$ identity matrix. Show that for
        any $b \in \mathbb{R}^m$, the equation $Ax=b$ has a solution. Try explaining why the matrix $A$
        can not have more rows than columns.
        \begin{proof}
            Let $A$ be an $m \times n$ matrix s.t. $AD=I_m$ where $I_m$ is the $m \times m$ identity 
            matrix. Consider $A(Db)$ with $b \in \mathbb{R}^m$ then
            \begin{align*}
                A(Db) &= (AD)b  && \text{(Associative Property)}\\
                &= I_{m}b \\
                &= b
            \end{align*}
            Thus $x=Db$ is a solution $Ax=b$.
        \end{proof}
        Since $Ax=b$ has a sultion $\forall b \in \mathbb{R}^m$ A has a pivot in every row. Since each column 
        has to have a pivot there cannot be more rows than columns.
        \item Assume $A$ is an $m \times n$ matrix and there exists $n \times m$ matrices $C$ and $D$ such that
        $CA=I_n$ and $AD=I_m$. Prove that $m=n$ and $C=D$.
        \begin{proof}
            Let the matrices $A$ being $m \times n$, and $C,D$ being $n \times m$ s.t. $CA=I_n$ and 
            $AD=I_m$. Then by problem 1 since $CA=I_n$ we know that $m \geq n$ and by problem 2 
            since $AD=I_m$ we know that $n \geq m$. Thus $m=n$. Consider the product $CAD$, then
            \begin{align*}
                CAD &= (CA)D && \text{(Associative Property)} \\
                &=I_{n}D \\
                &=D
            \end{align*} But we can also do this as
            \begin{align*}
                CAD &= C(AD) && \text{(Associative Property)} \\
                &=CI_m \\
                &=C
            \end{align*}
            $\therefore C=D$.
        \end{proof}
    \end{enumerate}

\end{document}