\documentclass[12pt]{article}

\usepackage{fancyhdr}
\usepackage{extramarks}
\usepackage{amsmath}
\usepackage{amsthm}
\usepackage{amsfonts}
\usepackage{amssymb}
\usepackage{tikz}
\usepackage[plain]{algorithm}
\usepackage{algpseudocode}
\usepackage{fancyhdr}
\usepackage{array}
\usepackage{wrapfig}
\usepackage{adjustbox}
\usepackage{enumitem}


\newlength{\strutheight}
\settoheight{\strutheight}{\strut}
\newtheorem{theorem}{Theorem}
\renewcommand\qedsymbol{$\blacksquare$}

\topmargin=-0.45in
\evensidemargin=0in
\oddsidemargin=0in
\textwidth=6.5in
\textheight=9.0in
\headsep=0.25in


\title{Abstract Algebra Homework 1}
\author{Zachary Meyner}
\date{}

\begin{document}
\setlength\extrarowheight{3pt}
\maketitle
\begin{enumerate}
    \item Which of the following multiplication tables defined on the set $G$ = $\{a,b,c,d\}$ form a
          group? Support your answer in each case.
    \begin{theorem}
        Let $a,b,c \in G$. Given $a$ is the Identity element of the set $G$, \\
        then \(a \circ (b \circ c) = (a \circ b) \circ c\), \(\forall b,c \in G\).
    \end{theorem}
    \begin{proof}
        Let $d \in G$ with $b \circ c = d$. Then \(a \circ (b \circ c) \implies a \circ d = d\). \\
        We also have \((a \circ b) \circ c \implies b \circ c = d\). \\
        \(\therefore a \circ (b \circ c) = (a \circ b) \circ c\).
    \end{proof}
    \begin{enumerate}
        \item \begin{adjustbox}{valign=T,raise=\strutheight,minipage={1.0\linewidth}}
              \begin{wrapfigure}{l}{0pt}
              \begin{tabular}{c|cccc}
              $\circ$ & $a$ & $b$ & $c$ & $d$ \\
              \cline{1-5}
              $a$       & $a$ & $c$ & $d$ & $a$ \\
              $b$       & $b$ & $b$ & $c$ & $d$ \\
              $c$       & $c$ & $d$ & $a$ & $b$ \\
              $d$       & $d$ & $a$ & $b$ & $c$ \\

            \end{tabular}
            \end{wrapfigure}
            This Cayley Table does not form a group because it is not Associative: \\
            $a \circ (b \circ c) = d$ and \\
            $(a\circ b) \circ c = a$,  so \\
            $a \circ (b \circ c) \neq (a \circ b) \circ c$
            \end{adjustbox}
            \\ \\ \\
        \item \begin{adjustbox}{valign=T,raise=\strutheight,minipage={1.0\linewidth}}
              \begin{wrapfigure}{l}{0pt}
              \begin{tabular}{c|cccc}
              $\circ$ & $a$ & $b$ & $c$ & $d$ \\
              \cline{1-5}
              $a$       & $a$ & $b$ & $c$ & $d$ \\
              $b$       & $b$ & $a$ & $d$ & $c$ \\
              $c$       & $c$ & $d$ & $a$ & $b$ \\
              $d$       & $d$ & $c$ & $b$ & $a$ \\

            \end{tabular}
            \end{wrapfigure}
            Closure: Every element in the Cayley Table is in the set $G$, so it is closed. \\
            Identity: taking any element and multiplying it by $a$ returns that element. So $a$ is the indentity element. \\
            Inverse: A diagonal is formed in the table with the identity element $a$, so every element is its own inverse. \\
            Associative: Because $a$ is the identity element it is associative with every 
            \end{adjustbox}
    \end{enumerate}
\end{enumerate}
\end{document}