\documentclass[12pt]{article}

\usepackage{geometry}
\usepackage{fancyhdr}
\usepackage{extramarks}
\usepackage{amsmath}
\usepackage{amsthm}
\usepackage{amsfonts}
\usepackage{amssymb}
\usepackage{tikz}
\usepackage[plain]{algorithm}
\usepackage{algpseudocode}
\usepackage{fancyhdr}
\usepackage{array}
\usepackage{wrapfig}
\usepackage{adjustbox}
\usepackage{enumitem}


\newlength{\strutheight}
\settoheight{\strutheight}{\strut}
\newtheorem{theorem}{Theorem}
\renewcommand\qedsymbol{$\blacksquare$}
\newcommand\setitemnumber[1]{\setcounter{enumi}{\numexpr#1-1\relax}}

\geometry{letterpaper,portrait,margin=1in}


\title{\large Abstract Algebra Homework 4}
\author{\large Zachary Meyner}
\date{}

\begin{document}
\maketitle
\begin{enumerate}[label=\textbf{\arabic*}.]
	\setitemnumber{2}
	\item Suppose that $G$ is a finite group with 60 elements. What are the orders of possible subgroups of $G$? \\
	      By Lagrange's Theorem the orders of the possible subgroups must divide 60, so possible
	      orders are $\{1, 2, 3, 4, 5, 6, 10, 12, 15, 20, 30, 60\}$
	\setitemnumber{11}
	\item Let $H$ be a subgroup of a group $G$ and suppose that $g_1, g_2 \ in G$. Prove that the following 
	conditions are equivalent.
    \begin{enumerate}
        \item $g_1H = g_2H$
        \item $Hg_1^{-1} = Hg_2^{-1}$
        \item $g_1H \subseteq g_2H$
        \item $g_2 \in g_1H$
        \item $g_1^{-1}g_2 \in H$
    \end{enumerate}
    \begin{proof} (a) $\Rightarrow$ (c) Done in class \\
(c) $\Rightarrow$ (d) Because $g_1H \subseteq g_2H \ \exists h \in H$ s.t. $g_1 = g_2h$. Then $g_1h^{-1} = g_2$. So $g_2 \in g_1H$. \\
(d) $\Rightarrow$ (e) Done in class \\
(e) $\Rightarrow$ (b) Because $g_1^{-1}g_2 \in H \ \exists h \in H$ s.t. $g_1^{-1}g_2 = h \implies g_1^{-1} = hg_2^{-1}$ and $h^{-1}g_1^{-1} = g_2^{-1}$. \\
Case 1: Suppose $h_1g_1^{-1} \in Hg_1^{-1}$. Then $h_1g_1^{-1} = h_1hg_2^{-1} \in Hg_2^{-1}$. \\
$\therefore Hg_1^{-1} \subseteq Hg_2^{-1}$. \\
Case 2: Suppose $h_2g_2^{-1} \in Hg_2^{-1}$. Then $h_2g_2^{-1} = h_2h^{-1}g_1^{-1} \in Hg_1^{-1}$. \\
$\therefore Hg_2^{-1} \subseteq Hg_1^{-1}$ and $Hg_1^{-1} = Hg_2^{-1}$. \\
(b) $\Rightarrow$ (a) Since $Hg_1^{-1} = Hg_2^{-1} \ \exists h \in H$ s.t. $g_1^{-1} = hg_2^{-1} \implies g_2h^{-1} = g_1$ and $g_2 = g_1h$. \\
Case 1: Let $g_2h_1 \in g_2H$. Then $g_2h_1 = g_1hh_1 \in g_1H$. \\
$\therefore g_2H \subseteq g_1H$. \\
Case 2: Let $g_1h_2 \in g_1H$. Then $g_1h_2 = g_2h^{-1}h_2 \in g_2H$. \\
$\therefore g_1H \subseteq g_2H$ and $g_1H = g_2H$.
    \end{proof}
\end{enumerate}

\end{document}