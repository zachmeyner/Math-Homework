\documentclass[12pt]{article}

\usepackage{fancyhdr}
\usepackage{extramarks}
\usepackage{amsmath}
\usepackage{amsthm}
\usepackage{amsfonts}
\usepackage{amssymb}
\usepackage{tikz}
\usepackage[plain]{algorithm}
\usepackage{algpseudocode}
\usepackage{fancyhdr}
\usepackage{array}
\usepackage{wrapfig}
\usepackage{adjustbox}
\usepackage{enumitem}


\newlength{\strutheight}
\settoheight{\strutheight}{\strut}
\newtheorem{theorem}{Theorem}
\renewcommand\qedsymbol{$\blacksquare$}
\newcommand\setitemnumber[1]{\setcounter{enumi}{\numexpr#1-1\relax}}

\topmargin=-0.45in
\evensidemargin=0in
\oddsidemargin=0in
\textwidth=6.5in
\textheight=9.0in
\headsep=0.25in


\title{\large Abstract Algebra Homework 1}
\author{\large Zachary Meyner}
\date{}

\begin{document}
\maketitle
\begin{enumerate}[label=\textbf{\arabic*}.]
  \setitemnumber{2}
    \item Which of the following multiplication tables defined on the set $G$ = $\{a,b,c,d\}$ form a
          group? Support your answer in each case.
    \begin{theorem}
        Let $a,b,c \in G$. Given $a$ is the Identity element of the set $G$, \\
        then \(a \circ (b \circ c) = (a \circ b) \circ c\), \(\forall b,c \in G\).
    \end{theorem}
    \begin{proof}
        Let $d \in G$ with $b \circ c = d$. Then \(a \circ (b \circ c) \implies a \circ d = d\). \\
        We also have \((a \circ b) \circ c \implies b \circ c = d\). \\
        \(\therefore a \circ (b \circ c) = (a \circ b) \circ c\).
    \end{proof}
    \begin{enumerate}
        \item \begin{adjustbox}{valign=T,raise=\strutheight,minipage={1.0\linewidth}}
              \begin{wrapfigure}{l}{0pt}
              \begin{tabular}{c|cccc}
              $\circ$ & $a$ & $b$ & $c$ & $d$ \\
              \cline{1-5}
              $a$       & $a$ & $c$ & $d$ & $a$ \\
              $b$       & $b$ & $b$ & $c$ & $d$ \\
              $c$       & $c$ & $d$ & $a$ & $b$ \\
              $d$       & $d$ & $a$ & $b$ & $c$ \\

            \end{tabular}
            \end{wrapfigure}
            This Cayley Table does not form a group because it is not Associative: 
            \begin{align*}
              a \circ (b \circ c) &= d \ \text{and} \\
              (a \circ b) \circ c &= a \\
              \implies a \circ (b \circ c) &\neq (a \circ b) \circ c
            \end{align*}
            \end{adjustbox}
            \\ \\ \\
        \item \begin{adjustbox}{valign=T,raise=\strutheight,minipage={1.0\linewidth}}
              \begin{wrapfigure}{l}{0pt}
              \begin{tabular}{c|cccc}
              $\circ$ & $a$ & $b$ & $c$ & $d$ \\
              \cline{1-5}
              $a$       & $a$ & $b$ & $c$ & $d$ \\
              $b$       & $b$ & $a$ & $d$ & $c$ \\
              $c$       & $c$ & $d$ & $a$ & $b$ \\
              $d$       & $d$ & $c$ & $b$ & $a$ \\

            \end{tabular}
            \end{wrapfigure}
            Closure: Every element in the Cayley Table is in the set $G$, so it is closed. \\
            Identity: taking any element and multiplying it by $a$ returns that element. So $a$ is the indentity element. \\
            Inverse: A diagonal is formed in the table with the identity element $a$, so every element is its own inverse. \\
            Associative: Because $a$ is the identity element it is associative with every set of two elements by Theorem 1. 
            Because every element $p_{ij} = p_{ji}$ it is commutative as well, so only one permutation 
            of the elements $b,c,d$ needs to be tested for associativity. We have \\ 
            \((b \circ c) \circ d = a\), and \(b \circ (c \circ d) = a\), so 
            \((b \circ c) \circ d = b \circ (c \circ d)\). \\
            Therefore this Cayley Table is a group.
            \end{adjustbox}
            \pagebreak
        \item \begin{adjustbox}{valign=T,raise=\strutheight,minipage={1.0\linewidth}}
              \begin{wrapfigure}{l}{0pt}
              \begin{tabular}{c|cccc}
              $\circ$ & $a$ & $b$ & $c$ & $d$ \\
              \cline{1-5}
              $a$       & $a$ & $b$ & $c$ & $d$ \\
              $b$       & $b$ & $c$ & $d$ & $a$ \\
              $c$       & $c$ & $d$ & $a$ & $b$ \\
              $d$       & $d$ & $a$ & $b$ & $c$ \\
            \end{tabular}
            \end{wrapfigure}
            This Cayley Table is the same as the Cayley Table for the group $(\mathbb{Z}_4 , +)$ 
            where \(a = 0, \ b = 1, \ c = 2, \ d = 3\), so This Cayley Table must be a group.
            \end{adjustbox}
            \\ \\ \\ \\ \\
        \item \begin{adjustbox}{valign=T,raise=\strutheight,minipage={1.0\linewidth}}
              \begin{wrapfigure}{l}{0pt}
              \begin{tabular}{c|cccc}
              $\circ$ & $a$ & $b$ & $c$ & $d$ \\
              \cline{1-5}
              $a$       & $a$ & $b$ & $c$ & $d$ \\
              $b$       & $b$ & $a$ & $c$ & $d$ \\
              $c$       & $c$ & $b$ & $a$ & $d$ \\
              $d$       & $d$ & $d$ & $b$ & $c$ \\
            \end{tabular}
            \end{wrapfigure}
            The identity element of this Cayley Table is $a$. 
            There is no inverse for $d$ where $d \circ p = a$ in this Cayley Table.
            Therefore this Cayley Table is not a Group.
            \end{adjustbox}
            \\ \\ \\ \\ \\
    \end{enumerate}
      \setitemnumber{13}    
        \item Show that \(\mathbb{R}^* = \mathbb{R}\setminus \{0\} \) is a group under the operation of multiplication. \\
        Let $a,b,c \in \mathbb{R}^*$ \\
        Closure: Multiplying two real numbers will always return result in a real number. \\
        Associativity: The Field Axioms of real numbers state \(a*(b*c) = (a*b)*c\), so it is Associative as well. \\
        Identity: The multiplicative identity for $\mathbb{R}^*$ is $1$ because \(a \cdot 1=a \ \forall a \in \mathbb{R}^*\). \\
        Inverse: The multiplicative inverse for $\mathbb{R}^*$ is $\frac{1}{a}$ because 
        \(a \cdot \frac{1}{a} = \frac{a \cdot 1}{a} = \frac{a}{a} = 1 \ \forall a \in \mathbb{R}^*\).
      \setitemnumber{27}
        \item Prove that the inverse of \(g_{1}g_{2} \dots g_{n}\) is \(g_{n}^{-1}g_{n-1}^{-1} \dots g_{1}^{-1}\).
        \begin{proof}[Proof by induction]
          Base Case $(n = 1)$:
          \[g_{1}g_{1}^{-1} = e\]
          Inductive Hypothesis $(n = k)$:
          \[(g_{1}g_{2} \dots g_{k})(g_{k}^{-1}g_{k-1}^{-1} \dots g_{1}^{-1}) = e \]
          Inductive Step $(n = k + 1)$ \quad (WTS: \((g_{1}g_{2} \dots g_{k + 1})(g_{k+1}^{-1}g_{k}^{-1} \dots g_{1}^{-1}) = e \) )
          \begin{align*}
          &\ (g_{1}g_{2} \dots g_{k + 1})(g_{k+1}^{-1}g_{k}^{-1} \dots g_{1}^{-1}) \\
          =&\ (g_{1}g_{2} \dots g_{k})(g_{k + 1})(g_{k+1}^{-1})(g_{k}^{-1}g_{k-1}^{-1} \dots g_{1}^{-1}) && \text{(Associative Property)}\\
          =&\ (g_{1}g_{2} \dots g_{k})(e)(g_{k}^{-1}g_{k-1}^{-1} \dots g_{1}^{-1}) && \text{(Definition)} \\
          =&\ (g_{1}g_{2} \dots g_{k})(g_{k}^{-1}g_{k-1}^{-1} \dots g_{1}^{-1}) && \text{(Definition)} \\
          =&\ e && \text{(Inductive HypothesisS )}
          \end{align*}
        \end{proof}
        \pagebreak
        \setitemnumber{31}
        \item Show that if $a^2 = e$ for all elements in $a$ in a group $G$, then $G$ must be abelian.
        \begin{proof}
          Let $a,b \in G$. Consider $a \cdot b \cdot b \cdot a$. Since $a^2 = a \cdot a = e \ \forall a \in G$, and $G$ is a Group, we have
          \begin{align*}
            a \cdot b \cdot b \cdot a &= a \cdot (b \cdot b) \cdot a && \text{(Commutative Property)} \\
            &= a \cdot e \cdot a && \text{(Hypothesis)} \\
            &= a \cdot a && \text{(Identity Property)} \\
            &= e && \text{(Hypothesis)} 
          \end{align*}
          But $a \cdot b$ has an inverse element of $a \cdot b$, and we have proven that $(a \cdot b) \cdot (b \cdot a) = e$, 
          so $a \cdot b = b \cdot a \ \forall a,b \in G$. Therefore $G$ is commutative which makes it an Abelian Group.
        \end{proof}
\end{enumerate}
\end{document}