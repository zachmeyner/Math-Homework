\documentclass[12pt]{article}

\usepackage{geometry}
\usepackage{fancyhdr}
\usepackage{extramarks}
\usepackage{amsmath}
\usepackage{amsthm}
\usepackage{amsfonts}
\usepackage{amssymb}
\usepackage{tikz}
\usepackage[plain]{algorithm}
\usepackage{algpseudocode}
\usepackage{fancyhdr}
\usepackage{array}
\usepackage{wrapfig}
\usepackage{adjustbox}
\usepackage{enumitem}


\newlength{\strutheight}
\settoheight{\strutheight}{\strut}
\newtheorem{theorem}{Theorem}
\renewcommand\qedsymbol{$\blacksquare$}
\newcommand\setitemnumber[1]{\setcounter{enumi}{\numexpr#1-- -1\relax}}

\geometry{letterpaper,portrait,margin=1in}


\title{\large Historical Roots of Mathematics Homework 1}
\author{\large Zachary Meyner}
\date{}

\begin{document}
\maketitle
\begin{enumerate}[label=\arabic*.]
    \setitemnumber{1}
    \item Write the follwing problems in hieroglyphics and then perform the addition:
    \begin{enumerate}[label=\alph*.]
        \item $46+23$ \vspace{50mm}
        \item $64+28$ \vspace{50mm}
        \item $4297+1351$ \vspace{50mm}
    \end{enumerate}
    \item 
    \begin{enumerate}[label=\alph*.]
        \item Show that 
        \[ \frac{2}{n} = \frac{1}{3n}+\frac{5}{3n} \]
        hence that $2/n$ can be expressed as a sum of unit fractions whenever $n$ is divisible by 5. \\
        That is 
        If $n=5k \ k \in \mathbb{Z}$, then $\displaystyle\frac{2}{n} = \displaystyle\frac{1}{3n} + \displaystyle\frac{1}{3k}$.
        \begin{proof}
            Let $n,k \in \mathbb{Z}$ s.t. $n = 5k$. Then
            \begin{align*}
                \frac{2}{n} &= \frac{6}{3n} \\
                &= \frac{1}{3n} + \frac{5}{3n} \\
                &= \frac{1}{3n} + \frac{5}{3(5k)} \\
                &= \frac{1}{3n} + \frac{1}{3k}
            \end{align*}
        \end{proof}
        \item Note: $25 = 5(5)$
        \[\frac{2}{25} = \frac{1}{75} + \frac{1}{15} \]
        Note: $65 = 5(13)$
        \[\frac{2}{65} = \frac{1}{195} + \frac{1}{39} \]
        Note: $85 = 5(17)$
        \[\frac{2}{85} = \frac{1}{255} + \frac{1}{51} \] 
    \end{enumerate}
\end{enumerate}

\end{document}