\documentclass[12pt]{article}

\usepackage{geometry}
\usepackage{fancyhdr}
\usepackage{extramarks}
\usepackage{amsmath}
\usepackage{amsthm}
\usepackage{amsfonts}
\usepackage{amssymb}
\usepackage{tikz}
\usepackage[plain]{algorithm}
\usepackage{algpseudocode}
\usepackage{fancyhdr}
\usepackage{array}
\usepackage{wrapfig}
\usepackage{adjustbox}
\usepackage{enumitem}


\newlength{\strutheight}
\settoheight{\strutheight}{\strut}
\newtheorem{theorem}{Theorem}
\renewcommand\qedsymbol{$\blacksquare$}
\newcommand\setitemnumber[1]{\setcounter{enumi}{\numexpr#1-1\relax}}

\geometry{letterpaper,portrait,margin=1in}


\title{\large Intro to Analysis Homework 2}
\author{\large Zachary Meyner}
\date{}

\begin{document}
\maketitle
\begin{enumerate}
	\item Let $S$ be a nonempty bounded set in $\mathbb{R}$. Let $b<0$ and consider
	      $bS=\{bs:s \in S\}$. Prove $\sup(bS) = b * \inf(S)$
	      \begin{proof}
		      Let $\sup(S) = s$ We know that $\sup(bS) = b\sup(S) = bs$.  So $\forall s_0 \in S \ s_0 \leq s$.
		      Becsue $b < 0$ multiplying it into the inequality give is $bs_0 \geq bs \ \forall bs_0 \in bS$. So by
		      definition $bs$ is the smallest element in $bS$ when $b < 0$, so $bs = \inf(bS) = b*\inf(S)$.
	      \end{proof}
	\item Let $I_n = \Big[1, 1 + \frac{1}{n} \Big] \ \forall n \in \mathbb{N}$. Prove $\bigcap_{n=1}^{\infty} I_n = \{1\}$.
	      \begin{proof}
		      Clearly 1 is in $\Big[1, 1 + \frac{1}{n} \Big]$. BMOC Let $x \in \bigcap_{n=1}^{\infty}I_n$. Then
		      \begin{gather*}
			      1 < x \leq 1 + \frac{1}{n} \\
			      0 < x-1 \leq \frac{1}{n} \\
		      \end{gather*}
		      Since $x-1 > 0$ by Archimedean Property $\exists m \in \mathbb{N}$ s.t.
		      \begin{align*}
			      x-1 > \frac{1}{m} \\
			      \implies x > 1+\frac{1}{m}
		      \end{align*}
		      but $x < 1+ \frac{1}{n} \ \forall n \in \mathbb{N}$. $\therefore \bigcap_{n=1}^{\infty}I_n = \{1\}$.
	      \end{proof}
	\item Consider the set $S = \Big\{\frac{1}{n}-\frac{1}{m}: n,m \in \mathbb{N} \Big\}$. Find the infimum and supremum of the
	      set. Then, prove your assertions. \\
	      $\inf(S) = -1$
	      \begin{proof}
		      (i) First we'll show -1 is a lower bound on $S$. Let $\frac{1}{n_0} - \frac{1}{m_0} \in S$ Then when
		      $n \geq 1$ $\frac{1}{m_0} \leq 1 \implies \frac{-1}{m_0} \geq -1$ Then $\frac{1}{n_0} - \frac{1}{m_0} \geq \frac{1}{n_0} - 1$. \\
		      Also $n>0$ so $\frac{1}{n_0} > 0$, which means $\frac{1}{n_0} - 1 > 0 -1$. \\
		      Together this means $\frac{1}{n_0} - \frac{1}{m_0} \geq \frac{1}{n_0} - 1 > 0-1 = -1$. So $\frac{1}{n_0} - \frac{1}{m_0} > - 1$. So -1 is a
		      lower bound on $S$. \\
		      (ii) BMOC suppose $x \in \mathbb{R}$ s.t. $x$ is the greatest lower bound on $S$. \\
		      So $x > -1$ and $x \leq s \ \forall s \in S$. Then $x + 1 > 0$. and by Archimedean Property $\exists m_0 \in \mathbb{N}$ s.t. $\frac{1}{m_0} < x+1 \implies \frac{1}{m_0} - 1 < x$
		      but $\frac{1}{m_0} - 1 \in S$, and $x \leq s \ \forall s \in S$, a contradiction. Thus there is no $x$ that exists.
	      \end{proof}
	      $\sup(S) = 1$
	      \begin{proof}
		      (i) First we'll show 1 is an upper bound on $S$. Let $\frac{1}{n_0} - \frac{1}{m_0} \in S$.
		      Since $n \geq 1$ and $\frac{1}{n} \leq 1$ we know $\frac{1}{n_0} \leq 1$ and $\frac{1}{n_0} - \frac{1}{m_0} \leq 1 - \frac{1}{m_0}$. \\
		      Now with $n>0$ we have $\frac{1}{n} > 0$. This means $\frac{1}{m_0} > 0 \implies \frac{-1}{m_0} < 0$. \\
		      With both of these we have $\frac{1}{n_0} - \frac{1}{m_0} \leq 1 - \frac{1}{m_0} < 1 - 0 = 1$. \\
		      (ii) BMOC suppose $x \in \mathbb{R}$ s.t. $x$ is the least upper bound on $S$. \\
		      So $x < 1 \implies 1-x > 0$. By the Archimedean Property $\exists n_0 \in \mathbb{N}$ s.t. \\
		      $\frac{1}{n_0} < 1-x \implies \frac{1}{n_0} - 1 < -x \implies 1- \frac{1}{n_0} > x$, but $1- \frac{1}{n_0} \in S$, so $x$ is not an
		      upper bound on $S$, a contradiction. \\ $\therefore \sup(S) = 1$.
	      \end{proof}
	\item Let $I_n = \Big(2-\frac{1}{n},2\Big) \ \forall n \in \mathbb{N}$. Prove $\bigcap_{n=1}^{\infty}I_n = \emptyset$.
	      \begin{proof}
		      BMOC let $x \in \bigcap_{n=1}^{\infty}I_n$. Then
		      \begin{gather*}
			      2-\frac{1}{n} < x < 2 \\
			      2-\frac{1}{n} < x-2 < 0 \\
			      0 < 2-x < \frac{1}{n} - 2 \\
		      \end{gather*}
		      Since $2-x > 0$ by Archimedean Property $\exists m \in \mathbb{N}$ s.t.
		      \begin{align*}
			      2-x > \frac{1}{m}           \\
			      \implies -x > \frac{1}{m}-2 \\
			      \implies x < 2-\frac{1}{m}
		      \end{align*}
		      but $x > 2-\frac{1}{n} \ \forall n \in \mathbb{N}$. $\therefore \bigcap_{n=1}^{\infty}I_n = \emptyset$
	      \end{proof}
	\item Find the infimum of the set and prove your result.
	      \begin{equation*}
		      S = \bigg\{\frac{3+n}{n}:n \in \mathbb{N}\bigg\}
	      \end{equation*}
	      $\inf(S) = 1$
		  \begin{proof}
			  (i) $3+n \geq n \implies \frac{3+n}{n} \geq 1$ so 1 is clearly a lower bound on $S$. \\
			  (ii) BMOC suppose $x \in \mathbb{R}$ s.t. $x > 1$. So $x-1 > 0 \implies \frac{x-1}{3} > 0$. \\
			  By Archimedean Property $\exists m \in \mathbb{N}$ s.t. $\frac{x-1}{3} > \frac{1}{m} \implies x-1 > \frac{3}{m} \implies x > 1+\frac{3}{m} \\
			  \implies x > \frac{3+m}{m}$ but $x$ is a lower bound on $\frac{3+n}{n}$ a contradiction. \\
			  $\therefore \inf(S) = 1$
		  \end{proof}
	\item Find the limits of the following sequences. Prove your assertions.
	      \begin{enumerate}
		      \item \[\lim_{n \to \infty} \frac{2n-1}{n^2+2}. = 0\]
		            \begin{proof}
			            Let $\varepsilon > 0$ be given, consider $k = \frac{2}{\varepsilon}$ \\
			            Then $\forall n \geq k$
			            \begin{align*}
				            \bigg|\frac{2n-1}{n^2+2} - 0\bigg| & = \bigg|\frac{2n-1}{n^2+2}\bigg| \\
				                                               & = \frac{2n-1}{n^2+2}             \\
				                                               & < \frac{2n}{n^2}                 \\
				                                               & = \frac{2}{n}                    \\
				                                               & \leq \varepsilon
			            \end{align*}
			            ($\frac{2}{k} = \varepsilon \implies \frac{2}{\varepsilon} = k $)
		            \end{proof}
		      \item \[\lim_{n \to \infty} \frac{n^2-n}{n^2+2} = 1\]
		            \begin{proof}
			            Let $\varepsilon > 0$ be given, consider $k = \frac{3}{\varepsilon} $ \\
			            Then $\forall n \geq k$
			            \begin{align*}
				            \bigg|\frac{n^2-n}{n^2+2} - 1\bigg| & = \bigg|\frac{n^2-n}{n^2+2} - \frac{n^2+2}{n^2+2}\bigg| \\
				                                                & = \bigg|\frac{-n-2}{n^2+2}\bigg|                        \\
				                                                & = \bigg|\frac{-(n+2)}{n^2+2}\bigg|                      \\
				                                                & = \frac{n+2}{n^2+2}                                     \\
				                                                & < \frac{n+2n}{n^2}                                      \\
				                                                & \leq \frac{3n}{n^2}                                     \\
				                                                & = \frac{3}{n}                                           \\
				                                                & \leq \varepsilon
			            \end{align*}
			            ($\frac{3}{k} = \varepsilon \implies k = \frac{3}{\varepsilon}$)
		            \end{proof}
		      \item \[\lim_{n \to \infty} \frac{(-1)^n\sqrt{n}}{2n-1} = 0\]
		            \begin{proof}
			            Let $\varepsilon > 0$ be given, consider $k = \frac{1}{\varepsilon^2}$ \\
			            Then $\forall n \geq k$
			            \begin{align*}
				            \bigg|\frac{(-1)^n\sqrt{n}}{2n-1} - 0\bigg| & = \bigg|\frac{(-1)^n\sqrt{n}}{2n-1} \bigg| \\
				                                                        & = \frac{\sqrt{n}}{2n-1}                    \\
				                                                        & < \frac{\sqrt{n}}{2n-n}                    \\
				                                                        & = \frac{1}{\sqrt{n}}                       \\
				                                                        & \leq \varepsilon
			            \end{align*}
			            ($\frac{1}{\sqrt{k}} = \varepsilon \implies \sqrt{k} = \frac{1}{\varepsilon} \implies k = \frac{1}{\varepsilon^2}$)
		            \end{proof}
		      \item \[\lim_{n \to \infty} \frac{n^2-n}{n^3-2n-4} = 0\]
		            \begin{proof}
			            Let $\varepsilon > 0$ be given, consider $k = \max(3, \frac{3}{\varepsilon}) $ \\
			            Then $\forall n \geq k$
			            \begin{align*}
				            \bigg|\frac{n^2-n}{n^3-2n-4} - 0\bigg| & = \bigg|\frac{n^2-n}{n^3-2n-4}\bigg|                                 \\
				                                                   & = \frac{n^2-n}{|n^3-2n-4|}                                           \\
				                                                   & = \frac{n^2-n}{n^3-2n-4}             &  & (n \geq 3)                 \\
				                                                   & = \frac{n^2}{n^3-2n-4}                                               \\
				                                                   & < \frac{n^2}{\frac{n^3}{3}}          &  & (n^3-2n-4 > \frac{n^3}{3}) \\
				                                                   & = \frac{1}{\frac{n}{3}}                                              \\
				                                                   & = \frac{3}{n}                                                        \\
				                                                   & \leq \varepsilon                                                     \\
			            \end{align*}
			            ($\frac{3}{k} = \varepsilon \implies k = \frac{3}{\varepsilon} $)
		            \end{proof}
	      \end{enumerate}
\end{enumerate}
\end{document}