\documentclass[12pt]{article}

\usepackage{geometry}
\usepackage{fancyhdr}
\usepackage{extramarks}
\usepackage{amsmath}
\usepackage{amsthm}
\usepackage{amsfonts}
\usepackage{amssymb}
\usepackage{tikz}
\usepackage[plain]{algorithm}
\usepackage{algpseudocode}
\usepackage{fancyhdr}
\usepackage{array}
\usepackage{wrapfig}
\usepackage{adjustbox}
\usepackage{enumitem}


\newlength{\strutheight}
\settoheight{\strutheight}{\strut}
\newtheorem{theorem}{Theorem}
\renewcommand\qedsymbol{$\blacksquare$}
\newcommand\setitemnumber[1]{\setcounter{enumi}{\numexpr#1--1\relax}}

\geometry{letterpaper,portrait,margin=1in}


\title{\large Intro to Analysis Homework 8}
\author{\large Zachary Meyner}
\date{}

\begin{document}
\maketitle
\begin{enumerate}
    \item Suppose $f$ is continuous in $[a,b]$ and $f(c) > 0$ for some $c \in (a,b)$. Then show that 
    there exists an open interval for which $f(x) >0$ on the interval.
    \begin{proof}
        Consider $\varepsilon = \frac{f(x)}{2} > 0$. Since $f$ is continuous at $x=c, \ \exists \delta > 0$ s.t. $\forall x \in [a,b]$ 
        with $0 < |x-c| < \delta$ with $|f(x) - f(c)| < \varepsilon = \frac{f(c)}{2}$. Consider the interval $(c-\delta, c + \delta)$.
        If $x \in (c-\delta, c+\delta)$
        \begin{align*}
            f(x) &= f(c)-f(c)+f(x) \\
            &= f(c) - (f(c)+f(x)) \\
            &\geq f(c) - |f(c)+f(x)| \\
            &\geq f(c) - \frac{f(c)}{2} \\
            &= \frac{f(c)}{2} > 0
        \end{align*}
    \end{proof}
    \item Show that $f(x) = x^3-2x^2-3x+1$ has at least one zero on $[-1, 1]$ \\
    $f(1) = 3$ and $f(-1) = -5$, so by the Location of Roots Theorem $\exists c$ in the domain s.t. 
    $f(c) = 0$.
    
    \item Let $f: \mathbb{R} \to \mathbb{R}$ be continuous function and suppose $f(x) \leq 0$ for all rational numbers. 
    Prove that $f(x) \leq 0$ for all real numbers.
    \begin{proof}
        BMOC, suppose $\exists c \in \mathbb{R}$ s.t. $f(c) > 0$. Then by number 1 there is an open 
        interval $(a,b)$ for which $f(x) > 0 \ \forall x \in (a,b)$ with $c \in (a,b)$, but $(a,b)$ is a nonempty, 
        open interval, so $\exists d \in (a,b)$ s.t. $d \in \mathbb{Q}$. Thus $f(d) > 0$, this is a contraction, so $c$ does 
        not exists.
    \end{proof}

    \item Consider $f(x) = \begin{cases}
                                3x^2+2 &x \neq 2 \\
                                0 &x =2
                            \end{cases}$.
    Prove that $f$ is not continuous at $x=2$. \\
    Consider $(x_n) = 2 + \frac{1}{n}$. Clearly $\lim\limits_{n \to \infty} x_n = 2$. We know $(x_n) \neq 2 \ \forall n$ \\
    So $f(x_n) = 3(x_n)^2 + 2$, so $\lim\limits_{n \to \infty}f(x_n) = 3(x_n)+2 = 3{(2)}^{2}+2 = 14$.  \\
    But $f(\lim\limits_{n \to \infty} x_n) = f(2) = 0$. $14 \neq 0$ so this function is not continuous at $x=2$.

    \item Suppose $f,g: \mathbb{R} \to \mathbb{R}$ are both continuous functions. Show that $3f -2g$ is a continuous 
    function too.
    \begin{proof}
        Let $\varepsilon > 0$ be given, and let $c \in \mathbb{R}$. \\
        Since $f$ is continuous at $c, \ \exists \delta_f >0$ s.t. $\forall x \in \mathbb{R}$ with $0 < |x-a| < \delta_f$ 
        \[\big|f(x) - f(a)\big| < \frac{\varepsilon}{6}\]
        Since $g$ is continuous at $c, \ \exists \delta_g >0$ s.t. $\forall x \in \mathbb{R}$ with $0 < |x-a| < \delta_g$ 
        \[\big|g(x) - g(a)\big| < \frac{\varepsilon}{4}\]
        Consider $\delta = \min(\delta_f, delta_g)$ \\
        Then $\forall x \in \mathbb{R}$ with $0 < |x-a| < \delta$
        \begin{align*}
            \big|(3f(x)-2g(x)) - (3f(a)-2g(a))\big| &= \big| 3f(x)-3f(a) + 2g(a)+2g(x) \big| \\
            &\leq 3|f(x) - f(a)| + 2|g(a) - g(x)| \\
            &< \frac{3\varepsilon}{6} + \frac{2\varepsilon}{4} \\
            &= \varepsilon 
        \end{align*}
    \end{proof}

    \item For each statement, conclude if the statement is true because of Bolzano's Intermediate 
    Value Theorem
    \begin{enumerate}
        \item If $f$ is a function satisfying that $f(0) = 10$ and $f(8) = 2$, then it must be that 
        $\exists x \in (0, 8)$ satisfying that $f(x) = 5$. \\
        We cannot conclude that $\exists x \in (0,8)$ satisfying $f(x) = 5$ because the function could be discontinuous.

        \item Suppose revenue can be modeled by a function $R(x)$, where x is measured in 
        thousands of units for units ranging from 0 to 20,000. Revenue for 2,000 units is 
        known to be \$4500 and revenue for 3,000 units is known to be \$7200, but at no 
        point in time was revenue \$5500. We can conclude therefore that the revenue 
        function is not continuous. \\
        If $R$ is continuous then $\exists c$ where $R(c) = 5500$.

        \item If $f$ is a continuous function satisfying that $f(20) = 10$ and $f(8) = 2$, then \\ 
        $\exists x \in (8, 20)$ satisfying $f(x) = 15$. \\
        We can not make a conclustion on this because the known values for $f(x)$ when 
        $x \in (8, 20)$ only guarantee that $f(x)$ touches all the values in a range of at least 
        $(2, 10)$.
    \end{enumerate}
\end{enumerate}


\end{document}