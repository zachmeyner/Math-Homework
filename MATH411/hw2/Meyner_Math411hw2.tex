\documentclass[12pt]{article}

\usepackage{geometry}
\usepackage{fancyhdr}
\usepackage{extramarks}
\usepackage{amsmath}
\usepackage{amsthm}
\usepackage{amsfonts}
\usepackage{amssymb}
\usepackage{tikz}
\usepackage[plain]{algorithm}
\usepackage{algpseudocode}
\usepackage{fancyhdr}
\usepackage{array}
\usepackage{wrapfig}
\usepackage{adjustbox}
\usepackage{enumitem}


\newlength{\strutheight}
\settoheight{\strutheight}{\strut}
\newtheorem{theorem}{Theorem}
\renewcommand\qedsymbol{$\blacksquare$}
\newcommand\setitemnumber[1]{\setcounter{enumi}{\numexpr#1-1\relax}}

\geometry{letterpaper,portrait,margin=1in}


\title{\large Intro to Analysis Homework 2}
\author{\large Zachary Meyner}
\date{}

\begin{document}
\maketitle
\begin{enumerate}
	\item Let $S$ be a nonempty bounded set in $\mathbb{R}$. Let $b<0$ and consider
	      $bS=\{bs:s \in S\}$. Prove $\sup(bS) = b * \inf(S)$
	      \begin{proof}
		      Let $\sup(S) = s$ We know that $\sup(bS) = b\sup(S) = bs$.  So $\forall s_0 \in S \ s_0 \leq s$.
		      Becsue $b < 0$ multiplying it into the inequality give is $bs_0 \geq bs \ \forall bs_0 \in bS$. So by
		      definition $bs$ is the smallest element in $bS$ when $b < 0$, so $bs = \inf(bS) = b*\inf(S)$.
	      \end{proof}
	\item Let $I_n = \Big[1, 1 + \frac{1}{n} \Big] \ \forall n \in \mathbb{N}$. Prove $\bigcap_{n=1}^{\infty} I_n = \{1\}$.
	      \begin{proof}
		      Clearly 1 is in $\Big[1, 1 + \frac{1}{n} \Big]$. BMOC Let $x \in \bigcap_{n=1}^{\infty}I_n$. Then
		      \begin{gather*}
			      1 < x \leq 1 + \frac{1}{n} \\
			      0 < x-1 \leq \frac{1}{n} \\
		      \end{gather*}
		      Since $x-1 > 0$ by Archimedean Property $\exists m \in \mathbb{N}$ s.t.
		      \begin{align*}
			      x-1 > \frac{1}{m} \\
			      \implies x > 1+\frac{1}{m}
		      \end{align*}
		      but $x < 1+ \frac{1}{n} \ \forall n \in \mathbb{N}$. $\therefore \bigcap_{n=1}^{\infty}I_n = \{1\}$.
	      \end{proof}
	\item Consider the set $S = \Big\{\frac{1}{n}-\frac{1}{m}: n,m \in \mathbb{N} \Big\}$. Find the infimum and supremum of the
	      set. Then, prove your assertions. \\
	      $\inf(S) = -1$ and $\sup(S) = 1$
	\item Let $I_n = \Big(2-\frac{1}{n},2\Big) \ \forall n \in \mathbb{N}$. Prove $\bigcap_{n=1}^{\infty}I_n = \emptyset$.
	      \begin{proof}
            BMOC let $x \in \bigcap_{n=1}^{\infty}I_n$. Then
            \begin{gather*}
                2-\frac{1}{n} < x < 2 \\
                2-\frac{1}{n} < x-2 < 0 \\
                0 < 2-x < \frac{1}{n} - 2 \\
            \end{gather*}
            Since $2-x > 0$ by Archimedean Property $\exists m \in \mathbb{N}$ s.t.
            \begin{align*}
                2-x > \frac{1}{m} \\
                \implies -x > \frac{1}{m}-2 \\
                \implies x < 2-\frac{1}{m}
            \end{align*}
            but $x > 2-\frac{1}{n} \ \forall n \in \mathbb{N}$. $\therefore \bigcap_{n=1}^{\infty}I_n = \emptyset$
	      \end{proof}
    \item Find the infimum of the set and prove your result.
          \begin{equation*}
              S = \bigg\{\frac{3+n}{n}:n \in \mathbb{N}\bigg\}
          \end{equation*}
\end{enumerate}

\end{document}